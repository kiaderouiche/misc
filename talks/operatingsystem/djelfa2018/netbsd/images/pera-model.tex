% The Purdue Enterprise Reference Architecture (PERA) model.
% Author: Erno Pentzin (2013)
\documentclass{article}
\usepackage{tikz}
%%%<
\usepackage{verbatim}
\usepackage[active,tightpage]{preview}
\PreviewEnvironment{tikzpicture}
\setlength\PreviewBorder{10pt}%
%%%>
\begin{comment}
:Title: Purdue Enterprise Reference Architecture Model
:Tags: Chains;Networking
:Author: Erno Pentzin
:Slug: pera-model

Adapted from the figure in the book by Macaulay and Singer (2011).
\end{comment}
\usetikzlibrary{chains}
\begin{document}

\begin{tikzpicture}[
	scale=0.75,
	start chain=1 going below, 
	start chain=2 going right,
	node distance=1mm,
	desc/.style={
		scale=0.75,
		on chain=2,
		rectangle,
		rounded corners,
		draw=black, 
		very thick,
		text centered,
		text width=8cm,
		minimum height=12mm,
		fill=blue!30
		},
	it/.style={
		fill=blue!10
	},
	level/.style={
		scale=0.75,
		on chain=1,
		minimum height=12mm,
		text width=2cm,
		text centered
	},
	every node/.style={font=\sffamily}
]

% Levels
\node [level] (Level 5) {Level 5};
\node [level] (Level 4) {Level 4};
\node [level] (Level 3) {Level 3};
\node [level] (Level 2) {Level 2};
\node [level] (Level 1.5) { };
\node [level] (Level 1) {Level 1};
\node [level] (Level 0) {Level 0};

% Descriptions
\chainin (Level 5); % Start right of Level 5
% IT levels
\node [desc, it] (Archives) {Archives/File Servers};
\node [desc, it, continue chain=going below] (ERP) {ERP/Finance/Messaging};
% ICS levels
\node [desc] (Operations) {Operations Management/Historians};
\node [desc] (Supervisory) {Supervisory Controls};
\node [desc, text width=3.5cm, xshift=2.25cm] (PLC) {PLC/RTU IP Communication};
\node [desc, text width=3.5cm, xshift=-4.5cm] (SIS) {Safety Instrumented Systems};
\node [desc, xshift=2.25cm] (IO) {I/O from Sensors};

\end{tikzpicture}

\end{document}
