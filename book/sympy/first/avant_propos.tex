\section{Avant-Propos}
Ce livre traite de SymPy, une bibliothèque de calcul symbolique entièrement écrite en Python un langage 
de programmation de haut niveau, orienté objet, totalement libre, conçu pour produire 
du code de qualité, portable et facile à intégrer. Ainsi la conception d'un programme scientifique  ou 
symbolique avec SymPy et Python est très rapide et offre au développeur une bonne productivité. En tant 
que bibliothèque pythonienne elle repose sur un langage dynamique, très souple d'utilisation et 
constitue un complément idéal à des langages compilés. Elle reste une bibliothèque complète et autosuffisant, pour des petits scripts fonctionnels de quelques lignes, comme pour des applicatifs complexes de plusieurs centaines de modules.

\subsection*{Pourquoi ce livre ?}
Il n'existe pas beaucoup d'ouvrages qui traitent du calcul symbolique en générale par  
rapport aux calculs numériques ou des ouvrages consacré aux bibliothèques symbolique écrite en Python 
est en particulier gravitent autour de SymPy mis à part un livre de 50 pages, quelques chapitres ou des 
lignes de codes cité à titre d'exemples. Citons le livre de référence de Svein Linge et Hans Petter 
Langtangen Programming for Computations – Python A Gentle Introduction to Numerical Simulations with 
Python, aux éditions Springer, ou encore une version du livre de 50 pages Instant SymPy Starter de Ronan 
Lamy, aux éditions Packt Publishing Limited, Le livre est Instant SymPy Starter de Ronan Lamy, c'est un 
guide de démarrage rapide, La documentation en ligne de SymPy est bonne, mais il serait plus facile de 
commencer avec ce livre. Alors, pourquoi ce livre ?

Si ce livre présente comme celui de Ronan Lamy les notions de la bibliothèque, celui-la ajoute des  
exemples originaux, des choix dans la présentation des classes, et une approche globale particulière et 
détaillé, il tente également d’ajouter à ce socle des éléments qui participent de la philosophie de la 
programmation en Python scientifique, aller plus loin dans le développement non scientifique, mettre en 
valeur L'intérêt et l'importance, à savoir :
\begin{itemize}
 \item des conventions de codage ;
 \item combiné l'approche symbolique et numérique;
 \item des bonnes pratiques de programmation et des techniques d’optimisation ;
\end{itemize}

Même si chacun de ces sujets pourrait à lui seul donner matière à des ouvrages entiers, les réunir dans 
un seul et même livre contribue à fournir une vue complète de ce qu’un développeur d'application 
scientifique en particulier et Python averti et son chef de projet mettent en œuvre quotidiennement.

\subsection*{A qui s'adresse l'ouvrage?}
Cet ouvrage s’adresse bien sûr aux développeurs de tous horizons mais également aux
étudiants,chercheurs, enseignants et chefs de projets. Ils ne trouveront pas dans ce livre de bases de 
programmation; une pratique minimale préalable est indispensable de Python, quel que soit le langage 
utilisé. Il n'est pour autant pas nécessaire de maîtriser la programmation orientée objet et 
la connaissance d'un langage impératif est suffisante.
Les développeurs Python débutants – ou les développeurs avertis ne connaissant pas
encore cette bibliothèque – trouveront dans cet ouvrage des techniques et sujets avancées, les patterns 
efficaces et l'application de certains design patterns objet, topologie, théorie des catégories, machine
learning.
Les étudiants et enseignants trouveront un ouvrage ouvert sur l'apprentissage par l'exercice résolus et 
une interprétation exercices mathématiques  
les chercheurs trouveront un outil léger et efficace à travers des approches poussées liées aux 
questions récentes en connections avec les mathématiques pures, appliquées et la physique théorique.
Les chefs de projets trouveront des éléments pratiques pour augmenter l’efficacité de
leurs équipes pluridisciplinaires, notamment la présentation des principaux modules à la fois issues de la bibliothèque standard, graphique et numérique.
\subsection*{Les objectifs des exemples et exercices Mathématiques dans ce livre}
Les objectifs d’un tel cours sont de montrer le caractère vivant des mathématiques
et leur omniprésence dans le développement des technologies, et d’initier l’étudiant au
processus de modélisation conduisant au développement de certaines applications des
mathématiques.
Quoique quelques-uns des sujets couverts sortent maintenant du cadre strict des
technologies, nous espérons faire comprendre que, oui, les mathématiques servent, et
elles servent dans plusieurs applications de la vie de tous les jours. De plus, certains
des sujets abordés sont en plein développement et permettent donc aux étudiants de se
rendre compte, souvent pour la première fois, que les mathématiques sont en évolution
et que de nombreuses questions demeurent ouvertes.
Puisque le cours accueille un nombre important de futurs maîtres au secondaire, il est
important de souligner que le but n'est pas de leur fournir des exemples d’applications
qu'ils pourront enseigner directement à leurs élèves, mais bien de leur présenter des
exemples tangibles d'applications et de leur donner des outils pour qu’ils puissent eux-
mêmes, plus tard, préparer des exemples d'applications à l'intention de leurs élèves. Ils
doivent sentir qu'ils enseigneront une matière d’une grande beauté, certes, mais dont
les applications ont façonné l'environnement humain et sa compréhension.
\subsection*{Le choix des sujets}
En choisissant les applications, nous avons porté une attention particulière aux points suivants:
\subsection*{Indication}
 \begin{enumerate}
  \item Quand un théorème est exposé ce dernier sera démontrais.
  \item Il sera question dans divers de suivre le schéma suivant:
  \begin{enumerate}
    \item Définition
    \item Indiquer le théorème si nécessaire mais sans démonstration ou donnée une démonstration.
    \item suivi d'exemple et une explication de l'implémentation au niveau des sources de SymPy 
  \end{enumerate}
 \end{enumerate}
