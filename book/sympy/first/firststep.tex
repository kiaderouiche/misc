\section{Calcul formel}
Le calcul formel, ou parfois calcul symbolique, est le domaine des mathématiques et de informatique qui s'intéresse aux algorithmes opérant sur des objets de nature mathématique par le biais de représentations finies et exactes. Ainsi, un nombre entier est représenté de manière finie et exacte par la suite des chiffres de son écriture en base 2. Étant données les représentations de deux nombres entiers, le calcul formel se pose par exemple la question de calculer celle de leur produit.

Le calcul formel est en général considéré comme un domaine distinct du calcul scientifique, cette dernière appellation faisant référence au calcul numérique approché à l'aide de nombres en virgule flottante, là où le calcul formel met l'accent sur les calculs exacts sur des expressions pouvant contenir des variables ou des nombres en précision arbitraire (en). Comme exemples d'opérations de calcul formel, on peut citer le calcul de dérivées ou de primitives, la simplification d'expressions, la décomposition en facteurs irréductibles de polynômes, la mise sous formes normales de matrices, ou encore la résolution des systèmes polynomiaux.

Sur le plan théorique, on s'attache en calcul formel à donner des algorithmes avec la démonstration qu'ils terminent en temps fini et la démonstration que le résultat est bien la représentation d'un objet mathématique défini préalablement. Autant que possible, on essaie de plus d'estimer la complexité des algorithmes que l'on décrit, c'est-à-dire le nombre total d'opérations élémentaires qu'ils effectuent. Cela permet d'avoir une idée a priori du temps d'exécution d'un algorithme, de comparer l'efficacité théorique de différents algorithmes ou encore éclairer la nature même du problème.
\subsection{Logiciel de système de calcul formel}
Dans cette section en va exposer les systèmes de calcul formel, leur intérêt qui à vue un renouveau ces dernières années à cause de l'émergence de technique, technologie et nouvelle approche de programmation pour le domaine scientifique et industriel, hormis le fait que le logiciel de calcul formel en soient sont un outil pédagogique 
indispensable pour les scientifiques et les ingénieurs

\begin{definition}
Un logiciel de système formel est un outil qui facilite le calcul symbolique. La partie principale de ce système est la manipulation des expressions mathématiques sous leur forme symbolique.
\end{definition}

\begin{example}
soit $G=\{x\in\mathbb{R}^2:|x|<3\}$ et noté par: $x^0=(1,1)$; en considère la fonction:
\begin{equation}
f(x)=\left\{\begin{aligned} & \mathrm{e}^{|x|} & & \text{si $|x-x^0|\leq 1/2$}\\
& 0 & & \text{si $|x-x^0|> 1/2$}\end{aligned}\right.
\end{equation}
The function $f$ has bounded support, we can take $A=\{x\in\mathbb{R}^2:|x-x^0|\leq 1/2+\epsilon\}$ for all $\epsilon\in\intoo{0}{5/2-\sqrt{2}}$.
\end{example}

cet exemple se traduit en forme symbolique avec la bibliothèque SymPy:

\subsection{Quelques logiciels de calcul formel}

qui exprime ce qui nous permet notre choix pour un CAS qui possède des caractéristiques techniques et sur le plan du coût très important quand peut résumer dans les points suivants:
\begin{enumerate}
	\item Leger. 
	\item S’appuie sur le langage de programmation Python.
	\item Portabilité dans toute transparence.
\end{enumerate}

L'un des systèmes qui peut nous permettre d'écrire cette exemple avec un ordinateurs avec SymPy qui semble mieux intégré
\subsection{SymPy vs Sagemath}
Peut être que parmi les CAS les plus proche de SymPy est sans aucun doute Sagemath[]
\subsection{Pourquoi choisir SymPy?}
