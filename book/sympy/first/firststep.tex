\section{Calcul symbolique avec SymPy}
Ce recueil d'exercices et de problèmes de programmation s'adresse aussi bien aux débutants qu'aux programmeurs confirmés. Il présente en effet plusieurs états d'esprit dont les deux principaux sont la programmation classique en Pascal pour les étudiants du premier cycle universitaire, et la programmation fonctionnelle en Lisp pour le second cycle.

Ce livre constitue un panorama (non exhaustif, mais suffisant) sur les langages de programmation, et offre une grande variété dans les sujets traités : graphiques, calcul matriciel, traitements de chaînes de caractères, graphes, intelligence artificielle...
\\
La première partie du livre sera consacré \`a la résolution par une approche symbolique au divers questions posées au étudiants et toute personnes qui aiment savoir et voir s'initier  pour des niveaux et des questions rencontrés, la deuxième partie du livre sera questions aux problèmes plus rencontrés pour des étudiants passionnée des questions entre mathématiques et technologies, chercheurs et développeurs d'applications scientifiques, la troisième partie plus consacré aux questions poussées

\subsection{Pourquoi programmer en symbolique }
Le symbolique est une grande importance d'un point de vue technique, car il permet
de limité les risques de bug dans l'exécution des programmes, dans le contexte de 
la vérification formelle si en prend le programme suivant:

\subsection{Calcul formel}

L'approche La simulation numérique est devenue essentielle dans de nombreux domaines tels que la mécanique des fluides et des solides, la météo, l'évolution du climat, la biologie ou les semi-conducteurs. Elle permet de comprendre, de prévoir, d'accéder là où les instruments de mesures s'arrêtent. 

Ce livre présente des méthodes performantes du calcul scientifique : matrices creuses, résolution efficace des grands systèmes linéaires, ainsi que de nombreuses applications à la résolution par éléments finis et différences finies. Alternant algorithmes et applications, les programmes sont directement présentés en langage C++. Ils sont sous forme concise et claire, et utilisent largement les notions de classe et de généricité du langage C++. 

Le contenu de ce livre a fait l'objet de cours de troisième année à l'école nationale supérieure d'informatique et de mathématiques appliquées de Grenoble (ENSIMAG) ainsi qu'au mystère de mathématiques appliquées de l'université Joseph Fourier. Des connaissances de base d'algèbre matricielle et de programmation sont recommandées. La maîtrise du contenu de cet ouvrage permet d'appréhender les principaux paradigmes de programmation du calcul scientifique. Il est alors possible d'appliquer ces paradigmes pour aborder des problèmes d'intérêt pratique, tels que la résolution des équations aux dérivées partielles, qui est abordée au cours de ce livre. La diversité des sujets abordés, l'efficacité des algorithmes présentés et leur écriture directe en langage C++ font de cet ouvrage un recueil fort utile dans la vie professionnelle d'un ingénieur. 

Le premier chapitre présente les bases fondamentales pour la suite : présentation du langage C++ à travers la conception d'une classe de quaternions et outils d'analyse asymptotique du temps de calcul des algorithmes. Le second chapitre aborde l'algorithme de transformée de Fourier rapide et développe deux applications à la discrétisation d'équations aux dérivées partielles par la méthode des différences finies. Le troisième chapitre est dédié aux matrices creuses et à l'algorithme du gradient conjugué. Ces notions sont appliquées à la méthode des éléments finis. En annexe sont groupés des exemples de génération de maillage et de visualisation graphique. 

S'il est cependant recommandé de maîtriser les notions du premier chapitre pour aborder le reste du livre, les chapitres deux et trois sont complètement indépendants et peuvent être abordés séparément. Ces chapitres sont complétés par des exercices qui en constituent des développements, ainsi que des notes bibliographiques retraçant l'historique des travaux et fournissant des références sur des logiciels et librairies récents implémentant ou étendant les algorithmes présentés. 

\subsection{Un peu de théorie}(source: Wikipédia)
Le calcul formel, ou parfois calcul symbolique, est le domaine des mathématiques et de informatique qui s'intéresse aux algorithmes opérant sur des objets de nature mathématique par le biais de représentations finies et exactes. Ainsi, un nombre entier est représenté de manière finie et exacte par la suite des chiffres de son écriture en base 2. Étant données les représentations de deux nombres entiers, le calcul formel se pose par exemple la question de calculer celle de leur produit.

Le calcul formel est en général considéré comme un domaine distinct du calcul scientifique, cette dernière appellation faisant référence au calcul numérique approché à l'aide de nombres en virgule flottante, là où le calcul formel met l'accent sur les calculs exacts sur des expressions pouvant contenir des variables ou des nombres en précision arbitraire (en). Comme exemples d'opérations de calcul formel, on peut citer le calcul de dérivées ou de primitives, la simplification d'expressions, la décomposition en facteurs irréductibles de polynômes, la mise sous formes normales de matrices, ou encore la résolution des systèmes polynomiaux.

Sur le plan théorique, on s'attache en calcul formel à donner des algorithmes avec la démonstration qu'ils terminent en temps fini et la démonstration que le résultat est bien la représentation d'un objet mathématique défini préalablement. Autant que possible, on essaie de plus d'estimer la complexité des algorithmes que l'on décrit, c'est-à-dire le nombre total d'opérations élémentaires qu'ils effectuent. Cela permet d'avoir une idée a priori du temps d'exécution d'un algorithme, de comparer l'efficacité théorique de différents algorithmes ou encore éclairer la nature même du problème.
\subsection{Logiciel de système de calcul formel}
Dans cette section en va exposer les systèmes de calcul formel, leur intérêt qui à vue un renouveau ces dernières années à cause de l'émergence de technique, technologie et nouvelle approche de programmation pour le domaine scientifique et industriel, hormis le fait que le logiciel de calcul formel en soient sont un outil pédagogique 
indispensable pour les scientifiques et les ingénieurs

\begin{definition}
Un logiciel de système formel est un outil qui facilite le calcul symbolique. La partie principale de ce système est la manipulation des expressions mathématiques sous leur forme symbolique.
\end{definition}

\begin{example}
soit $G=\{x\in\mathbb{R}^2:|x|<3\}$ et noté par: $x^0=(1,1)$; en considère la fonction:
\begin{equation}
f(x)=\left\{\begin{aligned} & \mathrm{e}^{|x|} & & \text{si $|x-x^0|\leq 1/2$}\\
& 0 & & \text{si $|x-x^0|> 1/2$}\end{aligned}\right.
\end{equation}
The function $f$ has bounded support, we can take $A=\{x\in\mathbb{R}^2:|x-x^0|\leq 1/2+\epsilon\}$ for all $\epsilon\in\intoo{0}{5/2-\sqrt{2}}$.
\end{example}

cet exemple se traduit en forme symbolique avec la bibliothèque SymPy:

\subsection{Quelques logiciels de calcul formel}

qui exprime ce qui nous permet notre choix pour un CAS qui possède des caractéristiques techniques et sur le plan du coût très important quand peut résumer dans les points suivants:
\begin{enumerate}
	\item Leger et 
	\item S’appuie sur le langage de programmation Python
	\item Portabilité dans toute transparence
\end{enumerate}

L'un des systèmes qui peut nous permettre d'écrire cette exemple avec un ordinateurs avec SymPy qui semble mieu intégré

\subsection{Bibliothèque SymPy}

Dans un cas plus simple l'exemple 1.1 se formule beaucoup plus dans un outil comme SymPy est une bibliothèque de calcul formel elle est aussi un environnement pour 
l’apprentissage de l’algèbre, l’analyse, géométrie, combinatoire, cryptographie, mécanique 
classique et quantique pour le lycée et l’université mais aussi un environnement de 
développement et de recherche. SymPy  écrit entièrement en Python un langage de 
programmation facile à apprendre et adapté à l’apprentissage,  elle fourni aux étudiant 
\textit{SymPyGamma} une application web   notamment des primitives générales de traitement des expressions algébriques (développement, factorisation, …), des aides à l’organisation des objets mathématiques intervenant dans la résolution d’un problème ainsi qu’une assistance à la preuve. Il permet au professeur de préparer et de suivre le travail de l’élève. Différentes maquettes ont été développées et testées auprès d’élèves. Dans la plus récente, nous nous sommes attachés à explorer une nouvelle forme d’activité algébrique. Alors que le calcul en papier crayon et les logiciels standards considèrent
 les expressions de façon isolée, l’environnement que nous développons organise en réseau les différentes expressions intervenant dans la résolution d’un problème. L’ordinateur peut facilement mettre à jour ce réseau quand l’utilisateur modifie certains de ses éléments. Il devient ainsi possible, pour aborder un problème générique, d’explorer facilement des cas particuliers et de conduire une généralisation. Les relations entre expressions algébriques sont mieux mises en évidence du fait de leur invariance dans les modifications du réseau. De façon très concise, Casyopée peut être défini
\subsubsection{SymPyGamma}
Est une interface onWeb marche avec un navigateur contient plusieurs catégorie liée de calcul, dynamique. L’Intérêt de cette outil qu'il est facilement partageable adapté pour l’enseignement et surtout l'auto-apprentissage 
\subsubsection{Besoin de rester dans le symbolique}
Le symbolique est une grande importance d'un point de vue technique, car il permet
de limité les risques de bug dans l'exécution des programmes, dans le contexte de 
la vérification formelle si en prend le programme suivant:

\subsubsection{Passage du symbolique au numérique}
Généralement, le symbolique parmi c'est  
\subsection{Faire des dessins}


