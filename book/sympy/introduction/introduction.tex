\section{Introduction}
Ce recueil d'exercices et de problèmes de programmation s'adresse aussi bien aux débutants qu'aux programmeurs confirmés. Il présente en effet plusieurs états d'esprit dont les deux principaux sont la programmation classique en Pascal pour les étudiants du premier cycle universitaire, et la programmation fonctionnelle en Lisp pour le second cycle.

Ce livre constitue un panorama (non exhaustif, mais suffisant) sur les langages de programmation, et offre une grande variété dans les sujets traités : graphiques, calcul matriciel, traitements de chaînes de caractères, graphes, intelligence artificielle...
\\
La première partie du livre sera consacré \`a la résolution par une approche symbolique au divers questions posées au étudiants et toute personnes qui aiment savoir et voir s'initier  pour des niveaux et des questions rencontrés, la deuxième partie du livre sera questions aux problèmes plus rencontrés pour des étudiants passionnée des questions entre mathématiques et technologies, chercheurs et développeurs d'applications scientifiques, la troisième partie plus consacré aux questions poussées

\subsection{Pourquoi programmer en symbolique }