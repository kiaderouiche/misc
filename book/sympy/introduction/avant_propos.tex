\section{Avant-Propos}
Ce livre traite de SymPy, une bibliothèque de calcul symbolique entièrement écrite en Python un langage 
de programmation de haut niveau, orienté objet, totalement libre, conçu pour produire 
du code de qualité, portable et facile à intégrer. Ainsi la conception d'un programme scientifique  ou 
symbolique avec SymPy et Python est très rapide et offre au développeur une bonne productivité. En tant 
que bibliothèque pythonienne elle repose sur un langage dynamique, très souple d'utilisation et 
constitue un complément idéal à des langages compilés. Elle reste une bibliothèque complète et autosuffisant, pour des petits scripts fonctionnels de quelques lignes, comme pour des applicatifs complexes de plusieurs centaines de modules.

\subsection*{Pourquoi ce livre ?}
Il n'existe pas beaucoup d'ouvrages qui traitent du calcul symbolique en générale par  
rapport aux calculs numériques ou des ouvrages consacré aux bibliothèques symbolique écrite en Python 
est en particulier gravitent autour de SymPy mis à part un livre de 50 pages, quelques chapitres ou des 
lignes de codes cité à titre d'exemples. Citons le livre de référence de Svein Linge et Hans Petter 
Langtangen Programming for Computations – Python A Gentle Introduction to Numerical Simulations with 
Python, aux éditions Springer, ou encore une version du livre de 50 pages Instant SymPy Starter de Ronan 
Lamy, aux éditions Packt Publishing Limited, Le livre est Instant SymPy Starter de Ronan Lamy, c'est un 
guide de démarrage rapide, La documentation en ligne de SymPy est bonne, mais il serait plus facile de 
commencer avec ce livre. Alors, pourquoi ce livre ?

Si ce livre présente comme celui de Ronan Lamy les notions de la bibliothèque, celui-la ajoute des  
exemples originaux, des choix dans la présentation des classes, et une approche globale particulière et 
détaillé, il tente également d’ajouter à ce socle des éléments qui participent de la philosophie de la 
programmation en Python scientifique, aller plus loin dans le développement non scientifique, mettre en 
valeur L’Intérêt et l'importance, à savoir :
\begin{itemize}
 \item des conventions de codage ;
 \item combiné l'approche symbolique et numérique;
 \item des bonnes pratiques de programmation et des techniques d’optimisation ;
\end{itemize}

Même si chacun de ces sujets pourrait à lui seul donner matière à des ouvrages entiers, les réunir dans 
un seul et même livre contribue à fournir une vue complète de ce qu’un développeur d'application 
scientifique en particulier et Python averti et son chef de projet mettent en œuvre quotidiennement.

\subsection*{A qui s'adresse l'ouvrage?}
Cet ouvrage s’adresse bien sûr aux développeurs de tous horizons mais également aux
étudiants,chercheurs, enseignants et chefs de projets. Ils ne trouveront pas dans ce livre de bases de 
programmation; une pratique minimale préalable est indispensable de Python, quel que soit le langage 
utilisé. Il n’est pour autant pas nécessaire de maîtriser la programmation orientée objet et 
la connaissance d’un langage impératif est suffisante.
Les développeurs Python débutants – ou les développeurs avertis ne connaissant pas
encore cette bibliothèque – trouveront dans cet ouvrage des techniques et sujets avancées, les patterns 
efficaces et l’application de certains design patterns objet, topologie, théorie des catégories, machine
learning.
Les étudiants et enseignants trouveront un ouvrage ouvert sur l'apprentissage par l'exercice résolus et 
une interprétation d’exercices mathématiques  
les chercheurs trouveront un outil léger et efficace à travers des approches poussées liées aux 
questions récentes en connections avec les mathématiques pures et appliquées de physique théorique.
Les chefs de projets trouveront des éléments pratiques pour augmenter l’efficacité de
leurs équipes pluridisciplinaires, notamment la présentation des principaux modules à la fois issues de  la bibliothèque standard, graphique et numérique.