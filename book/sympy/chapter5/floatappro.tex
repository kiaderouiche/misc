\chapter{Nombres à virgule flottante}\index{Nombres à virgule flottante}
Dans les chapitres suivants, les nombres à virgule flottante sont au cœur des
calculs ; il convient de les étudier car leur comportement suit des règles précises.
Comment représenter des nombres réels en machine ? Comme ces nombres ne
peuvent pas en général être codés avec une quantité finie d’information, ils ne
sont pas toujours représentables sur un ordinateur : il faut donc les approcher
avec une quantité de mémoire finie.
Un standard s’est dégagé autour d’une approximation des nombres réels avec
une quantité fixe d’information : la représentation à virgule flottante.
Dans ce chapitre, on trouve : une description sommaire des nombres à virgule
flottante et des différents types de ces nombres disponibles dans SymPy, et la 
démonstration de quelques-unes de leurs propriétés. Quelques exemples montreront
certaines des difficultés qu’on rencontre en calculant avec les nombres à virgule
flottante, quelques astuces pour arriver parfois à les contourner, en espérant 
développer chez le lecteur une prudence bien nécessaire ; en conclusion, nous essayons
de donner quelques propriétés que doivent posséder les méthodes numériques pour
pouvoir être utilisées avec ces nombres.