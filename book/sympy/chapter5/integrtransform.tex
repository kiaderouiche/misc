\chapter{Transformation d'intégrale}
 \section{Transformation de Fourier}
 \begin{exercise}
 \end{exercise}
 \begin{exercise}
 \end{exercise}
 \begin{exercise}
 \end{exercise}
  \subsection{Transformation de Fourier inverse}
  \begin{exercise}
 \end{exercise}
 \begin{exercise}
 \end{exercise}
 \section{Transformation de Laplace}
 \begin{python}
 \end{python}
 \begin{exercise}
 \end{exercise}
 \section{Transformation de Mellin}
 \begin{example}
  La transformée d'une distribution de Dirac ${\displaystyle \delta (x-a)}$, avec $a > 0$, est une fonction exponentielle ${\displaystyle s\mapsto a^{s-1}}$.
 \end{example}
 \begin{python}
 \end{python}
 \begin{example}
  La transformée de Mellin de la fonction ${\displaystyle f\,:\,x\mapsto \mathrm {H} (a-x)} {\displaystyle f\,:\,x\mapsto \mathrm {H} (a-x)}$, avec $a > 0$, est la fonction ${\displaystyle s\mapsto {\frac {a^{s}}{s}}} {\displaystyle s\mapsto {\frac {a^{s}}{s}}}$ sur le demi-plan $Re (s) > 0$
(où $H$ est la fonction de Heaviside, $f(x) = 1$ si $0 < x < a et f (x) = 0 si x > a$).
 \end{example}
 \begin{python}
 \end{python}
 \begin{example}
%   La transformée de Mellin de la fonction ${\frac {1}{\mathrm {e} ^{x}-1}}}$ est la fonction ${\displaystyle s\mapsto \Gamma (s)\zeta (s)}$ sur le demi-plan $Re (s) > 1$
 \end{example}
 \begin{python}
 \end{python}
 \subsection{Transformation de melin inverse}
 \section{Transfomation de Henkel}
 La transformée de Hankel (d'ordre zéro) est une transformée intégrale équivalente à une transformée de Fourier à deux dimensions avec un noyau intégral à symétrie radiale, appelée également transformation de Fourier-Bessel. Il est défini comme
 \[
 g \left(u, v \right)	= F_{r}\left[\left(r\right)\right]\left(u, v\right)
 \]
 \subsection{Transformation de Henkel inverse}
 
\chapter{Équations non linéaires}\index{Équations non linéaires}
Ce chapitre explique comment résoudre une équation non linéaire avec SymPy.
Dans un premier temps on étudie les équations polynomiales et on montre les
limitations de la recherche de solutions exactes. Ensuite on décrit le fonctionnement
de quelques méthodes classiques de résolution numérique. Au passage on indique
quels sont les algorithmes de résolution numérique implémentés dans SymPy.
Qu'est ce que non-linéaire et qu'est ce que une \'equation alg\'ebrique

Une \'equation alg\'ebrique est un polyn\^ome de la forme $P(x)$

\begin{equation}
\exp(-x)\sin(x) = \cos(x)
\end{equation}
\section{Équations algébriques}