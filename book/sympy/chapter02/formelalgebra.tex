\part{Algèbre et théorie des nombres}
\chapter{Anneaux et corps finis}
\section{Anneau des entiers modulo $n$}
\section{Corps finis}
\section{Probl\`emes élémentaires de théorie des nombres}
\subsection{Théorème de Wilson}\footnote{http://www.les-mathematiques.net/phorum/read.php?3,320294,320385, https://blogdemaths.wordpress.com/2015/10/26/le-theoreme-de-wilson/}
\'enoncé du théorème 
\begin{theorem}
 Un entier $p$ strictement plus grand que $1$ est un nombre premier si et seulement s'il divise $(p - 1)! + 1$, c'est-à-dire si et seulement si $(p-1)!+ 1 \equiv 0 (mod p)$
\end{theorem}

Alain Connes dans son papier "Autour du théorème de Wilson"\footnote{https://arxiv.org/abs/1809.02832}
\begin{remark}
Contrairement au petit théorème de Fermat, le théorème de Wilson est une condition nécessaire et suffisante pour tester la primalité. Toutefois, cela conduirait à un test très lent informatiquement, car le calcul de (p-1)! nécessite beaucoup d'opérations.
\end{remark}
\chapter{Polynômes}
\begin{exercise}
Considérons le polynôme $p(x) = a_{0} + a_{1} x + a_{2} x^{2} + a_{3} x^{3}$, où $a_{i} \neq 0$ $\forall i$. Le nombre minimum de multiplications nécessaires pour évaluer $p$ sur une entrée $x$ est:
\end{exercise}
\chapter{Algèbre linéaire}
Ce chapitre traite de l’algèbre linéaire exacte et symbolique, c’est-à-dire sur
des anneaux propres au calcul formel, tels que $Z$, des corps finis, des anneaux de
polynômes. Nous présentons les constructions sur les matrices et leurs espaces ainsi que les
opérations de base, puis les différents calculs possibles sur ces matrices, regroupés en deux thèmes : ceux liés à l’élimination de Gauss et aux transformations par équivalence à gauche, et ceux liés aux valeurs et espaces
propres et aux transformations de similitude.
