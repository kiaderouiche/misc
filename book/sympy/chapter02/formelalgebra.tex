\part{Algèbre et théorie des nombres}
\chapter{Anneaux et corps finis}
\section{Anneau des entiers modulo $n$}
\section{Corps finis}
\section{Quelques probl\`emes élémentaires de théorie des nombres}
\subsection{Théorème de Wilson}\footnote{\textbf{Note Historique.} Le théorème de Wilson a été découvert à la fin du dixième siècle par le mathématicien arabe Ibn al-Haytham $(965-1040)$. Le résultat ressurgit, sans démonstration, à la fin du dix-huitième siècle dans les écrits de Edward Waring qui l’attribue en 1770 à son élève John Wilson. L’année suivante, Lagrange en donne deux démonstrations dans son article [LAG]. En fait, Leibniz $(1646-1716)$ connaissait déjà le résultat et sa démonstration mais ne les avait pas publiés (voir [RAS] pour de plus amples considérations historiques).}
\'enoncé du théorème 
\begin{theorem}
 Un entier $p$ strictement plus grand que $1$ est un nombre premier si et seulement s'il divise $(p - 1)! + 1$, c'est-à-dire si et seulement si $(p-1)!+ 1 \equiv 0 (mod p)$
\end{theorem}
\textbf{Indication.}  Quatre démonstrations de ce résultat de Wilson. L'idée directrice des deux premières démonstrations est de remplacer ce calcul de congruence $(\mathbb{Z}/p\mathbb{Z})$ par un calcul dans $\mathbb{F}_{p}$ ce qui va permettre d'utiliser les propriétés d’un corps. L'idée de la troisième démonstration est d’utiliser les théorèmes de Sylow dans le groupe symétrique $\mathbb{\sigma}_{p}$, la quatrième est plutôt combinatoire qui repose sur l'identité algébrique $\Sigma_{i=0}^{n} (-1)^{i} C_n^{p}(x-i)^n = n!$ donnée en théorème l'auteur\footnote{https://arxiv.org/pdf/math/0406086.pdf} ainsi le théorème sera simplement comme
corollaire en remplaçant $n$ par $p-1$
\begin{proof}
A venir!.
\end{proof}
Alain Connes dans son article "Autour du théorème de Wilson"\footnote{Alain Connes, An essay on the Riemann Hypothesis, Open Problems in Mathematics.John Forbes Nash, Jr. Michael Th. Rassias Editors}, donne une approximation du nombre $\pi$ en somme de $sin$
\begin{remark}
Contrairement au petit théorème de Fermat, le théorème de Wilson est une condition nécessaire et suffisante pour tester la primalité. Toutefois, cela conduirait à un test très lent informatiquement, car le calcul de (p-1)! nécessite beaucoup d'opérations.
\end{remark}
\begin{python}
import sympy

def isPrime(n):
 if n == 4: return 
 return bool(math.factorial(n>>1)%n)
\end{python}
\chapter{Polynômes}
 \section{Anneaux et polynômes}
  \subsection{Introduction}
 Nous avons vu au chapitre 2 comment effectuer des calculs sur des expressions formelles, éléments de « l'anneau symbolique ». Dans ce chapitre en va manipuler le module dédié, \textcolor{red}{sympy.polys}, permettant de calculer des algèbres polynomiales sur divers domaines de coefficients implémentant un grand nombre de méthodes  allant d’outils simples comme la division polynomiale à des concepts avancés comprenant les bases de Gröbner et la factorisation multivariée sur des domaines de nombres algébrique:

\begin{exercise}
Considérons le polynôme $p(x) = a_{0} + a_{1} x + a_{2} x^{2} + a_{3} x^{3}$, où $a_{i} \neq 0$ $\forall i$. Le nombre minimum de multiplications nécessaires pour évaluer $p$ sur une entrée $x$ est:
\end{exercise}
\chapter{Algèbre linéaire}
Ce chapitre traite de l’algèbre linéaire exacte et symbolique, c’est-à-dire sur
des anneaux propres au calcul formel, tels que $Z$, des corps finis, des anneaux de
polynômes. Nous présentons les constructions sur les matrices et leurs espaces ainsi que les
opérations de base, puis les différents calculs possibles sur ces matrices, regroupés en deux thèmes : ceux liés à l’élimination de Gauss et aux transformations par équivalence à gauche, et ceux liés aux valeurs et espaces
propres et aux transformations de similitude.
