\part{Algèbre et théorie des nombres}
\chapter{Anneaux et corps finis}
\subsubsection{Anneau des entiers modulo $n$}
\subsubsection{Corps finis}
\chapter{Polynômes}
\begin{exercise}
Considérons le polynôme $p(x) = a_{0} + a_{1} x + a_{2} x^{2} + a_{3} x^{3}$, où $a_{i} \neq 0$ $\forall i$. Le nombre minimum de multiplications nécessaires pour évaluer $p$ sur une entrée $x$ est:
\end{exercise}
\chapter{Algèbre linéaire}
Ce chapitre traite de l’algèbre linéaire exacte et symbolique, c’est-à-dire sur
des anneaux propres au calcul formel, tels que $Z$, des corps finis, des anneaux de
polynômes. Nous présentons les constructions sur les matrices et leurs espaces ainsi que les
opérations de base, puis les différents calculs possibles sur ces matrices, regroupés en deux thèmes : ceux liés à l’élimination de Gauss et aux transformations par équivalence à gauche, et ceux liés aux valeurs et espaces
propres et aux transformations de similitude.
