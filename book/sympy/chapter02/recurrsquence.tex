\chapter{suites définies par une relation de récurrence}
Il y a pas de classe pour la construcution et le calcul de relation de récurrence de suite de fonction, le seul moyen est
de faire intervenir le module \textcolor{blue}{from.sympy.Function}
	\section{Suites définies par $u_{n+1} = f\left(u_{n} \right)$}
 \begin{definition}
 Considérons une suite définie par une relation $u_{n+1} = f\left(u_{n}\right)$ avec $u_{0} = a$. On peut définir la suite naturellement à l’aide d’un algorithme récursif. Prenons par exemple une suite logistique (suite définie par une récurrence
de la forme $x_{n+1} = rx_{n}\left(1 - x_{n} \right)$ :
 \end{definition}
 
 % http://math.univ-lille1.fr/~bodin/exo4/exohtml/p2node8.html
 \begin{exercise}
  \begin{enumerate}
   \item Étudier la suite définie par récurrence par $u_0=a$ et $ u_{n+1}= \cos u_n$, où $a$ est un nombre réel donné.
   \item Étudier la suite définie pour $ n\geqslant 1$ par $ u_n=\underbrace{\cos(\cos(\cos(\cdots(\cos}
_{n \; \rm fois\; cos} n)\cdots)))$
  \end{enumerate}
 \end{exercise}
	\section{Suites récurrentes linéaires}
  \begin{definition}
  Une suite récurrente linéaire est défini de $\mathbb{N} \longrightarrow \mathbb{R}$ est de la forme
  \[
  a_{k}u_{n+k}+ a_{k-1}u_{n+k-1} + a_{k-2}u_{n+k-2} + ...+a_{1}u_{n+1}+ a_{0}u_{0} = 0
  \]
  avec $\left(a\right)_{0\leq i \leq k}$ une famille de scalaire
  \begin{example}
  
  \end{example}
  \end{definition}
	\section{Suites récurrentes « avec second membre »}