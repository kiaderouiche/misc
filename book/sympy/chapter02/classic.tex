


\section{Programmation Orientée Objet}\index{Notations}
\begin{notation}
Given an open subset $G$ of $\mathbb{R}^n$, the set of functions $\varphi$ are:
\begin{enumerate}
\item Bounded support $G$;
\item Infinitely differentiable;
\end{enumerate}
a vector space is denoted by $\mathcal{D}(G)$. 
\end{notation}

 \subsection{POO}
\begin{exercise}
 Définir une classe Vecteur2D avec un constructeur fournissant les coordonnées par
défaut d’un vecteur du plan (par exemple : $x = 0$ et $y = 0$ ).
Dans le programme principal, instanciez un Vecteur2D sans paramètre, un Vecteur2D
avec ses deux paramètres, et affichez-les.
\end{exercise}
\begin{solution}
 en utilise le module sympy.geometry ce module fait appel à tout les outils et theories qui
 peuvents entre utiliser dans le cade de la géométrie dans le Plan.
 \begin{python}
 from sympy.geometry
  \end{python}
\end{solution}

\begin{exercise}
Enrichissez la classe Vecteur2D précédente en lui ajoutant une méthode d’affichage
et une méthode de surcharge d’addition de deux vecteurs du plan.
Dans le programme principal, instanciez deux Vecteur2D , affichez-les et affichez leur
somme.
\end{exercise}
\begin{solution}
\end{solution}

%------------------------------------------------
\subsection{Notions de COO et d’encapsulation}
%------------------------------------------------

%%
%\includeonly{geometry/euclid}
