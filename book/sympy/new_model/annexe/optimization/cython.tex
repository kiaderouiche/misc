\subsection{Cython}
Cython (http://www.cython.org/ ) est un métalangage qui permet de combiner du code
Python et des types de donn\'ees C, pour concevoir des extensions compilables pour
Python.
Dans un module Cython, il est possible de définir des variables C directement dans
le code Python et de définir des fonctions C qui prennent en paramètre des
variables C ou des objets Python.
Cython contr\^ole ensuite de manière transparente la génération de l’extension C, en
transformant le module en code C par le biais des API C de Python.
Toutes les fonctions Python du module sont alors automatiquement publiées.
Le gain de temps dans la conception introduit par Cython est considérable : toute la
mécanique habituellement mise en œuvre pour créer un module d’extension est
entièrement gérée par Cython.
Ainsi, la fonction max() du module calculs.c pr\'ec\'edemment présent\'ee devient :

Les fichiers Cython ont par convention l’extension pyx, en référence à l’ancien nom.

setup.py pour calculs.pyx

\begin{python}
from distutils.core import setup
from distutils.extension import Extension
from Cython.Distutils import build_ext

extension = Extension("calculs", ["calculs.pyx"])

setup(name="calculs", ext_modules=[extension],cmdclass={'build_ext': build_ext})

\end{python}
