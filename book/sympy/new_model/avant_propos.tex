\section{Avant-Propos}
Ce livre traite de Python, un langage de programmation de haut niveau, orienté objet,
totalement libre et terriblement efficace, conçu pour produire du code de qualité, 
portable et facile à intégrer. Ainsi la conception d'un programme Python est très rapide et
offre au développeur une bonne productivité. En tant que langage dynamique, il est
très souple d'utilisation et constitue un complément idéal à des langages compilés.
Il reste un langage complet et autosuffisant, pour des petits scripts fonctionnels de 
quelques lignes, comme pour des applicatifs complexes de plusieurs centaines de modules.

\subsection*{Pourquoi ce livre ?}
Il existe déjà de nombreux ouvrages excellents traduits de l'anglais qui traitent de
Python voire en présentent intégralité des modules disponibles. Citons Python en
concentré, le manuel de référence de Mark Lutz et David Ascher, aux éditions
O'Reilly, ou encore Apprendre à programmer avec Python de Gérard Swinnen, aux
éditions Eyrolles, inspiré en partie du texte How to think like a computer scientist
(Downey, Elkner, Meyers), et comme son titre l'indique, tr\'es p\'edadogique.
Alors, pourquoi ce livre ?