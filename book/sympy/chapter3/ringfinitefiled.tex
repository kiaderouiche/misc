\chapter{Anneaux et corps finis}
\section{Anneau des entiers modulo $n$}
\section{Corps finis}
\section{Quelques probl\`emes élémentaires de théorie des nombres}

\subsection{Théorème de Wilson}\footnote{\textbf{Note Historique.} Le théorème de Wilson a été découvert à la fin du dixième siècle par le mathématicien arabe Ibn al-Haytham $(965-1040)$. Le résultat ressurgit, sans démonstration, à la fin du dix-huitième siècle dans les écrits de Edward Waring qui l’attribue en 1770 à son élève John Wilson. L’année suivante, Lagrange en donne deux démonstrations dans son article [LAG]. En fait, Leibniz $(1646-1716)$ connaissait déjà le résultat et sa démonstration mais ne les avait pas publiés (voir [RAS] pour de plus amples considérations historiques).}
\'enoncé du théorème 
\begin{theorem}
 Un entier $p$ strictement plus grand que $1$ est un nombre premier si et seulement s'il divise $(p - 1)! + 1$, c'est-à-dire si et seulement si $(p-1)!+ 1 \equiv 0 (mod p)$
\end{theorem}
\textbf{Indication.}  Quatre démonstrations de ce résultat de Wilson. L'idée directrice des deux premières démonstrations est de remplacer ce calcul de congruence $(\mathbb{Z}/p\mathbb{Z})$ par un calcul dans $\mathbb{F}_{p}$ ce qui va permettre d'utiliser les propriétés d’un corps. L'idée de la troisième démonstration est d’utiliser les théorèmes de Sylow dans le groupe symétrique $\mathbb{\sigma}_{p}$, la quatrième est plutôt combinatoire qui repose sur l'identité algébrique $\Sigma_{i=0}^{n} (-1)^{i} C_n^{p}(x-i)^n = n!$ donnée en théorème l'auteur\footnote{https://arxiv.org/pdf/math/0406086.pdf} ainsi le théorème sera simplement comme
corollaire en remplaçant $n$ par $p-1$
\begin{proof}
A venir!.
\end{proof}
Alain Connes dans son article "Autour du théorème de Wilson"\footnote{Alain Connes, An essay on the Riemann Hypothesis, Open Problems in Mathematics.John Forbes Nash, Jr. Michael Th. Rassias Editors}, donne une approximation du nombre $\pi$ en somme de $sin$
\begin{remark}
Contrairement au petit théorème de Fermat, le théorème de Wilson est une condition nécessaire et suffisante pour tester la primalité. Toutefois, cela conduirait à un test très lent informatiquement, car le calcul de (p-1)! nécessite beaucoup d'opérations.
\end{remark}
\begin{python}
import sympy

def isPrime(n):
 if n == 4: return 
 return bool(math.factorial(n>>1)%n)
\end{python}
\subsection{Nombre p-adique}
Nous définissons les nombres p-adiques\footnote{Xavier Caruso, Computations with p-adic numbers. 2017}, discutons de leurs propriétés fondamentales et essayons d'expliquer, en sélectionnant quelques exemples pertinents, leur place dans la théorie des nombres, la géométrie algébrique et le calcul symbolique. La présentation ci-dessous est volontairement très résumée; nous renvoyons le lecteur intéressé à\footnote{Yvette Amice. Les nombres p-adiques. PUF (1975) et Fernando Gouvea. p-adic Numbers: An Introduction. Springer (1997) } pour un exposé plus complet de la théorie des nombres p-adiques.
\\
Les nombres p-adiques sont des objets très ambivalents auxquels on peut penser sous différents angles:
calcul, algébrique, analytique. Il s'avère que chaque point de vue conduit à sa propre définition
des nombres p-adiques: les informaticiens préfèrent souvent voir un nombre p-adique comme une séquence de
chiffres tandis que les algébristes préfèrent parler de limites projectives et les analystes sont plus à l'aise
avec des espaces et des finitions Banach. Bien sûr, toutes ces approches ont leur propre intérêt
et la compréhension des intersections entre eux est souvent la clé derrière le plus important
avances.
\begin{definition}
Un nombre $p-adique$
\end{definition}
