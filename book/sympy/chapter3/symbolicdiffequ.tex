\chapter{Équations différentielles}
 \section{Équations différentielles}
 \subsection{Introduction}
Si la méthode de George Pólya semble peu efficace, on peut faire appel à SymPy même si le domaine de la résolution formelle des équations différentielles demeure une faiblesse de nombreux logiciels de calcul. SymPy est en pleine évolution cependant et progresse à chaque version un peu plus en élargissant son spectre de
résolution.
\\
On peut, si on le souhaite, invoquer Sage afin d’obtenir une étude qualitative :
en effet, ses outils numériques et graphiques guideront l’intuition. C’est l’objet
de la section 14.2 du chapitre consacré au calcul numérique. Des outils d’étude
graphique des solutions sont donnés à la section 4.1.6. Des méthodes de résolution
à l’aide de séries se trouvent à la section 7.5.2.
On peut préférer résoudre les équations différentielles exactement. Sage peut
alors parfois y aider en donnant directement une réponse formelle comme nous le
verrons dans ce chapitre.
\\
Dans la plupart des cas, il faudra passer par une manipulation savante de
ces équations pour aider SymPy. Il faudra veiller à garder en tête que la solution
attendue d’une équation différentielle est une fonction dérivable sur un certain
intervalle mais que SymPy, lui, manipule des expressions sans domaine de définition.
La machine aura donc besoin d’une intervention humaine pour aller vers une
solution rigoureuse.
\\ 

Nous étudierons d’abord les généralités sur les équations différentielles ordi-
naires d’ordre 1 et quelques cas particuliers comme les équations linéaires, les
équations à variables séparables, les équations homogènes, une équation dépendant
d’un paramètre (§10.1.2) ; puis de manière plus sommaire les équations d’ordre 2
ainsi qu’un exemple d’équation aux dérivées partielles (§10.1.3). Nous terminerons
par l’utilisation de la transformée de Laplace (§10.1.4) et enfin la résolution de
certains systèmes différentiels (§10.1.5).
\\
On rappelle qu’une équation différentielle ordinaire (parfois notée EDO, ou
ODE en anglais) est une équation faisant intervenir une fonction (inconnue)
d’une seule variable, ainsi qu’une ou plusieurs dérivées, successives ou non, de la
fonction.
\\
Dans l’équation $y'(x) + x y\left(x\right) = e^{x}$ la fonction inconnue $y$ est appelée la
variable dépendante et la variable x (par rapport à laquelle $y$ varie) est appelée la
variable indépendante.
Une équation aux dérivées partielles (notée parfois EDP, ou PDE en anglais)
fait intervenir plusieurs variables indépendantes ainsi que les dérivées partielles
de la variable dépendante par rapport à ces variables indépendantes.
Sauf mention contraire, on considérera dans ce chapitre des fonctions d’une variable réelle.

\subsection{Équations différentielles ordinaires d’ordre 1}
\begin{definition}
Une équation différentielle ordinaire
\end{definition}

\begin{python}
from sympy import symbols, Function
x = symbols('x')
y = Function("y")( x)
\end{python}

\textbf{Équations du premier ordre pouvant être résolues directement par SymPy}. Nous allons étudier dans cette section comment résoudre avec SymPy Équations différentielles ordinaires d’ordre 1
les équations linéaires, les équations à variables séparables, les équations de Bernoulli, les équations homogènes, les équations exactes, ainsi que les équations de Riccati,
Lagrange et Clairaut.
\\
Équations linéaires. il s'agit d’équations du type:
  \[
  y'+ P(x)y = Q(x) 
  \] 
ou $P$ et $Q$ sont des fonctions continues sur des intervalles données.
\begin{example}
 \[
   y' + 3y = e^{x}
 \]
\end{example}
\subsection{ Équations d’ordre 2}
\textbf{Équations linéaires à coefficients constants.} Résolvons maintenant une équation du second ordre linéaire à coefficients constants, par exemple :
\[
 y''+3y = x^{2}-7x+31
\]
\subsection{Transformée de Laplace}
La transformée de Laplace permet de convertir une équation différentielle avec
des conditions initiales en équation algébrique et la transformée inverse permet
ensuite de revenir à la solution éventuelle de l’équation différentielle.
\\
Pour mémoire ensuite de revenir à la solution éventuelle de l’équation différentielle.
Pour mémoire, si $f$ est une fonction définie sur $\mathbb{R}$ en étant identiquement nulle
sur $\left]-\infty; 0\right[$, on appelle transformée de Laplace de f la fonction F définie, sous
certaines conditions, par :

\[
 \mathfrak{L}\left(f\left(x\right)\right) = F\left(s\right) = \int_{0}^{+\infty} e^{-sx} f\left(x\right)dx
\]
\subsection{ Systèmes différentiels linéaires}
\chapter{suites définies par une relation de récurrence}
Il y a pas de classe pour la construcution et le calcul de relation de récurrence de suite de fonction, le seul moyen est
de faire intervenir le module \textcolor{blue}{from.sympy.Function}
	\section{Suites définies par $u_{n+1} = f\left(u_{n} \right)$}
 \begin{definition}
 Considérons une suite définie par une relation $u_{n+1} = f\left(u_{n}\right)$ avec $u_{0} = a$. On peut définir la suite naturellement à l’aide d’un algorithme récursif. Prenons par exemple une suite logistique (suite définie par une récurrence
de la forme $x_{n+1} = rx_{n}\left(1 - x_{n} \right)$ :
 \end{definition}
 
 % http://math.univ-lille1.fr/~bodin/exo4/exohtml/p2node8.html
 \begin{exercise}
  \begin{enumerate}
   \item Étudier la suite définie par récurrence par $u_0=a$ et $ u_{n+1}= \cos u_n$, où $a$ est un nombre réel donné.
   \item Étudier la suite définie pour $ n\geqslant 1$ par $ u_n=\underbrace{\cos(\cos(\cos(\cdots(\cos}
_{n \; \rm fois\; cos} n)\cdots)))$
  \end{enumerate}
 \end{exercise}
	\section{Suites récurrentes linéaires}
  \begin{definition}
  Une suite récurrente linéaire est défini de $\mathbb{N} \longrightarrow \mathbb{R}$ est de la forme
  \[
  a_{k}u_{n+k}+ a_{k-1}u_{n+k-1} + a_{k-2}u_{n+k-2} + ...+a_{1}u_{n+1}+ a_{0}u_{0} = 0
  \]
  avec $\left(a\right)_{0\leq i \leq k}$ une famille de scalaire
  \begin{example}
  
  \end{example}
  \end{definition}
	\section{Suites récurrentes « avec second membre »}