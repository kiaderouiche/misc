\chapter{Polynômes}
 \section{Anneaux et polynômes}
  \subsection{Introduction}
 Nous avons vu au chapitre 2 comment effectuer des calculs sur des expressions formelles, éléments de « l'anneau symbolique ». Dans ce chapitre en va manipuler le module dédié, \textcolor{red}{sympy.polys}, permettant de calculer des algèbres polynomiales sur divers domaines de coefficients implémentant un grand nombre de méthodes  allant d’outils simples comme la division polynomiale à des concepts avancés comprenant les bases de Gröbner et la factorisation multivariée sur des domaines de nombres algébrique:
  \subsection{ Construction d’anneaux de polynômes}
En SymPy, les polynômes, comme beaucoup d’autres objets algébriques, sont en général à coefficients dans un anneau commutatif. C’est le point de vue que nous adoptons, mais la plupart de nos exemples concernent des polynômes sur un corps. Dans tout le chapitre, les lettres A et K désignent respectivement un anneau commutatif et un corps quelconques. La première étape pour mener un calcul dans une structure algébrique est souvent de construire R elle-même. On construit $\mathbb{Q}\left[x\right]$
 \section{Polynômes}
  \subsection{Création et arithmétique de base}
  \subsection{Vue d’ensemble des opérations sur les polynômes}
  \subsection{Changement d’anneau}\footnote{Contrairement à Sage, il peut y être que dans SymPy certain 
  fonctionnalité ne soi pas directement accessible que par passage à la programmation }
Changement d’anneau. La liste exacte des opérations disponibles, leur effet
et leur efficacité dépendent fortement de l’anneau de base. Par exemple, les
polynômes de $ZZ\left['x'\right]$ possèdent une méthode content qui renvoie leur contenu,
c’est-à-dire le pgcd de leurs coefficients ; ceux de $QQ\left['x'\right]$ non, l’opération étant
triviale. La méthode factor existe quant à elle pour tous les polynômes mais
déclenche une exception NotImplementedError pour un polynôme à coefficients
dans SR(le cas de Sage) ou dans $\mathbb{Z}/4\mathbb{Z}$. Par exemple Cette exception signifie que l’opération n’est pas disponible dans Sage pour ce type d’objet bien qu’elle ait un sens mathématiquement.
Il est donc très utile de pouvoir jongler avec les différents anneaux de coefficients
sur lesquels on peut considérer un « même » polynôme. Appliquée à un polynôme
de $A\left['x'\right]$, la méthode change\_ring renvoie son image dans $B\left[x\right]$, quand il y a une
façon naturelle de convertir les coefficients. La conversion est souvent donnée par
un morphisme canonique de A dans B : notamment, change\_ring sert à étendre
l’anneau de base pour disposer de propriétés algébriques supplémentaires. Ici par
exemple, le polynôme p est irréductible sur les entiers, mais se factorise sur R :
 \subsection{Itération}
 \subsection{Polyn\^ome irr\'eductible}
Le crit\`ere d'Eisenstein permet de prouver qu'un polyn\^ome est irr\'eductible dans $\mathbb{Q}\left[ X\right]$
 \section{Arithmétique euclidienne}
 \subsection{ Divisibilit\'e}
 \subsection{ Idéaux et quotients}
 \subsection{Idéaux}
 \section{ Factorisation et racines}
 \subsection{Factorisation}
 \subsection{ Recherche de racines}
 \subsection{ R\'esultant}
 \subsection{ Groupe de Galois}
 Par défaut le calcul de groupe de Galois n'est pas disponible dans SymPy, ce qui nous amènes encore
 une fois de programmer en ajoutant des modules, Le groupe de Galois d’un polynôme irréductible $p \in \mathbb{Q}\left[x\right]$ est un objet algébrique qui décrit certaines « symétries » des racines de $p$. Il s’agit d’un objet central de la théorie des équations algébriques. Notamment, l’équation $p\left(x\right) = 0$
est résoluble par radicaux, c’est-à-dire que ses solutions s’expriment à partir des coefficients de $p$ au moyen des quatre opérations et de l'extraction de racine n-ième, si et seulement si le groupe de Galois de $p$ est résoluble.
\section{ Fractions rationnelles}
 \subsection{ Construction et propriétés élémentaires}
 La division de deux polynômes (sur un anneau intègre) produit une fraction rationnelle. Son parent est le corps des fractions de l’anneau de polynômes, qui peut s’obtenir par :
 \begin{python}
 \end{python}
 On observe que la simplification n’est pas automatique. C’est parce que $RR$ est un anneau inexact, c’est-à-dire dont les éléments s’interprètent comme des approximations d’objets mathématiques. La méthode \textcolor{blue}{reduce} met la fraction sous forme réduite. Elle ne renvoie pas de nouvel objet, mais modifie la fraction rationnelle existante :
 \begin{python}
 \end{python}
Sur un anneau exact, en revanche, les fractions rationnelles sont automatiquement réduites. Les opérations sur les fractions rationnelles sont analogues à celles sur les polynômes. Celles qui ont un sens dans les deux cas (substitution, dérivée, factorisation...) s’utilisent de la même façon. Le tableau 7.6 énumère quelques autres
méthodes utiles. La décomposition en éléments simples et surtout la reconstruction
rationnelle méritent quelques explications.
 \subsection{Décomposition en éléments simples}
 Sage calcule la décomposition en éléments simples d’une fraction rationnelle $a \diagup b$
 \subsection{Reconstruction rationnelle}
 Un analogue de la reconstruction rationnelle présentée en §6.1.3 existe pour les
polynômes à coefficients dans $A = \mathbb{Z} \diagup n\mathbb{Z}$
 \section{Séries formelles}
 Une série formelle est une série en Groupes de matrices.tière vue comme une simple suite de coefficients, sans considération de convergence. Plus précisément, si $A$ est un anneau commutatif, on appelle séries formelles (en anglais formal  power series) d’indéterminée  à coefficients dans  les sommes formelles $\sum_{n=0}^{\infty} a_{n}x^{n}$ où ($a_{n}$) est une suite quelconque d’éléments de $A$. Munies des opérations d’addition et de multiplication naturelles
\[
 \sum_{n=0}^{\infty} a_{n}x^{n} + \sum_{n=0}^{\infty} b_{n}x^{n} = \sum_{n=0}^{\infty} \left(a_{n}+b_{n}\right) x^{n} 
 \],
\[
 \left(\sum_{n=0}^{\infty} a_{n}x^{n}\right) \left(\sum_{n=0}^{\infty} b_{n}x^{n}\right) =  \sum_{n=0}^{\infty} \left( \sum_{n=0}^{\infty} a_{i}b_{j}\right)x^{n}
\], les séries formelles forment un anneau noté $A\left[ \left[ x\right] \right] $.\\

les séries formelles forment un anneau noté Dans un système de calcul formel, ces séries sont utiles pour représenter des fonctions analytiques dont on n'a pas d’écriture exacte. Comme toujours, l’ordinateur fait les calculs, mais c’est à l'utilisateur de leur donner un sens mathématique. À lui par exemple de s’assurer que les séries qu’il manipule sont convergentes. 
 \subsection{Opérations sur les séries tronquées}
 \subsection{Développement de solutions d’équations}
 Face à une équation différentielle dont les solutions exactes sont trop compliquées à calculer ou à exploiter une fois calculées, ou tout simplement qui n’admet pas de solution en forme close, un recours fréquent consiste à chercher des solutions sous forme de séries. On commence habituellement par déterminer les solutions 
de l’équation dans l’espace des séries formelles, et si nécessaire, on conclut ensuite par un argument de convergence que les solutions formelles construites ont un sens analytique. SymPy peut être d’une aide précieuse pour la première étape. Considérons par exemple l’équation différentielle

\begin{example}
\[
 \left(x\right) = \sqrt{1+x^{2}}
\]
\end{example}

