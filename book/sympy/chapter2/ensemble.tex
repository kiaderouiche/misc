\chapter{Théorie des ensembles}\index{Théorie des ensemble}
\section{Limite de l'utilisation de set de la bibliothèque standard Python}\index{Un petit rappel pour les ensembles dans Python}
La bibliothèque standard dispose de la structure de donnée set, qui crée un ensemble $ens = ()$ ou $ens = {}$ de variable, et permet de le manipulez
\begin{python}
ens = set() ou ens 
\end{python}

ou écrire simplement 
\begin{python}
 ens = {1, 2, 4}
\end{python}
La notion d'objet immuable en Python est fondamentale,  une structure qui rappel les ensembles en mathématiques que soit fini ou infini est \textit{set}, importante, bien que dans le cadre de SymPy elle s'appuie entièrement sur Python avec certain modification, avec la collection d'objet.
\\

\textit{La fonction set accepte donc en argument un objet de type quelconque et s'efforce de le traduire dans un ensemble. Lorsqu'on ne passe aucun argument à set (option 2), ou qu'on lui passe une liste vide, set renvoie naturellement un ensemble vide; on aurait pu utiliser aussi bien, de la même manière, set(()), set({}), ou même set('') pour arriver au même résultat.}

\begin{exercise}
		Définir deux ensembles $X = \lbrace a, b, c, d\rbrace$ et  $Y = \lbrace s, b, d\rbrace$ , puis 			affichez les résultats suivants :
 		\begin{enumerate}
  			 \item les ensembles initiaux.
  			 \item le test d’appartenance de l’élément $c$ à $X$.
  			 \item le test d’appartenance de l’élément $a$ à $Y$.
  			 \item les ensembles $X - Y$ et $Y - X$.
  			 \item l’ensemble $X \cup Y$ (union).
  			 \item l'ensemble $X \cap Y$ (intersection).
	 \end{enumerate}
\end{exercise}

\begin{solution}
Il faut noter qu'il existe une solution qui se base sur le Python builtuints en utilisant la structure de donnée \textit{sets}. Mais comme en n'est dans la logique en utilise 
\begin{python}
from sympy import FiniteSet

X = FiniteSet('a', 'b', 'c', 'd')
Y = FiniteSet('s', 'b', 'd')

class MyClass(Yourclass):
    def __init__(self, my, yours):
        bla = '5 1 2 3 4'
        print bla
\end{python}
\begin{python}
class MyClass(Yourclass):
    def __init__(self, my, yours):
        bla = '5 1 2 3 4'
        print bla
\end{python}

\end{solution}
%%
\section{Produit cartésien}
 \begin{definition}
 Pour tout ensemble $A$ et tout $B$, il existe $P$ dont les éléments sont tous les couples dont la première composante appartient à $A$ et la seconde à $B$:
 \[
 \forall A \forall B
 \]
 
 \end{definition}
\section{L'algèbre de Boole comme ensemble}
 
 \begin{exercise}
 
 \end{exercise}
\section{Plan complexe}

\section{Mesure sur l'ensemble $\left[0,1\right] $ }
Premier exemple sur la mesure de l'ensemble  $\left[0,1\right]$
\begin{python}
In [1]: from sympy import Interval, Union
In [2]: Interval(0, 1).measure
Out[2]: 1
In [3]: Union(Interval(-0, 1), Interval(2, 3)).measure                          
Out[3]: 2
\end{python}