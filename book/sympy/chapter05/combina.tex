\part{Combinatoire}

Les sujets de ce chapitre sont du néanmoins axées sur des questions ou l'approche mathématique et 
physique et demandé 

\chapter{Dénombrement et combinatoire}\index{Dénombrement et combinatoire}
Ce chapitre aborde principalement le traitement avec SymPy des problèmes
combinatoires suivants : le dénombrement (combien y a-t-il d’éléments dans un
ensemble S ?), l’énumération (calculer tous les éléments de S, ou itérer parmi
eux), le tirage aléatoire (choisir au hasard un élément de S selon une loi, par
exemple uniforme). Ces questions interviennent naturellement dans les calculs de
probabilités (quelle est la probabilité au poker d’obtenir une suite, ou un carré
d’as ?), en physique statistique, mais aussi en calcul formel (nombre d’éléments
dans un corps fini), ou en analyse d’algorithmes. La combinatoire couvre un
domaine beaucoup plus vaste (ordres partiels, mots, théorie des représentations,
etc.) pour lesquels nous nous contentons de donner quelques pointeurs vers les
possibilités offertes par SymPy.
\chapter{Semi-groupe numérique}
\section{Introduction}
<<<<<<< HEAD
=======

>>>>>>> 6f5ba43eb8438b3098c6eebbfaa0a107a4a279e1
 
