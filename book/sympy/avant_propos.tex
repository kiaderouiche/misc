\section{Avant-Propos}
Ce livre traite de SymPy, une bibliothèque de calcul symbolique entièrement écrite en Python un langage de 
programmation de haut niveau, orienté objet, totalement libre et terriblement efficace, conçu pour produire 
du code de qualité, portable et facile à intégrer. Ainsi la conception d'un programme Python est très 
rapide et offre au développeur une bonne productivité. En tant que langage dynamique, il est
très souple d'utilisation et constitue un complément idéal à des langages compilés.
Il reste un langage complet et autosuffisant, pour des petits scripts fonctionnels de 
quelques lignes, comme pour des applicatifs complexes de plusieurs centaines de modules.

\subsection*{Pourquoi ce livre ?}
Il existe déjà de nombreux ouvrages excellents traduits de l'anglais qui traitent de
Python voire en présentent intégralité des modules disponibles. Citons Python en
concentré, le manuel de référence de Mark Lutz et David Ascher, aux éditions
O'Reilly, ou encore Apprendre à programmer avec Python de Gérard Swinnen, aux
éditions Eyrolles, inspiré en partie du texte How to think like a computer scientist
(Downey, Elkner, Meyers), et comme son titre l'indique, tr\'es p\'edadogique.
Alors, pourquoi ce livre ?

Si ce livre présente comme ses prédécesseurs les notions fondamentales du langage, avec
bien sûr des exemples originaux, des choix dans la présentation de certains modules, et
une approche globale particulière, il tente également d’ajouter à ce socle des éléments
qui participent de la philosophie de la programmation en Python, à savoir :
• des conventions de codage ;
• des recommandations pour la programmation dirigée par les tests ;
• des bonnes pratiques de programmation et des techniques d’optimisation ;
• des design patterns orientés objet.
Même si chacun de ces sujets pourrait à lui seul donner matière à des ouvrages
entiers, les réunir dans un seul et même livre contribue à fournir une vue complète de
ce qu’un développeur Python averti et son chef de projet mettent en œuvre quoti-
diennement.

\subsection*{A qui s'adresse l'ouvrage?}
Cet ouvrage s’adresse bien sûr aux développeurs de tous horizons mais également aux
chefs de projets.
Les développeurs ne trouveront pas dans ce livre de bases de programmation ; une
pratique minimale préalable est indispensable, quel que soit le langage utilisé. Il n’est
pour autant pas nécessaire de maîtriser la programmation orientée objet et la con-
naissance d’un langage impératif est suffisante.
Les développeurs Python débutants – ou les développeurs avertis ne connaissant pas
encore ce langage – trouveront dans cet ouvrage des techniques avancées, telles que la
programmation dirigée par les tests, les patterns efficaces et l’application de certains
design patterns objet.
Les chefs de projets trouveront des éléments pratiques pour augmenter l’efficacité de
leurs équipes, notamment la présentation des principaux modules de la bibliothèque
standard – pour lutter contre le syndrome du NIH (Not Invented Here) –, des con-
ventions de codage, et un guide explicite des techniques de programmation dirigée
par les tests.
