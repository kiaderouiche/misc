\section{Premier pas vers SymPy}
\subsection{Structure de données dans SymPy}

\subsection{Ensembles}
La notion d'objet immuable en Python est fondamentale,  une structure qui rappel les ensembles en mathématiques que soit fini ou infini est \textit{set}, importante, bien que
dans le cadre de SymPy elle s'appui entièrement sur Python avec certain modification, avec la collection d'objet.
\\

\textit{La fonction set accepte donc en argument un objet de type quelconque et s'efforce de le traduire dans un ensemble. Lorsqu'on ne passe aucun argument à set (option 2), ou qu'on lui passe une liste vide, set renvoie naturellement un ensemble vide; on aurait pu utiliser aussi bien, de la même manière, set(()), set({}), ou même set('') pour arriver au même résultat.}

\begin{exercise}
		Définir deux ensembles $X = \lbrace a, b, c, d\rbrace$ et  $Y = \lbrace s, b, d\rbrace$ , puis 			affichez les résultats suivants :
 		\begin{enumerate}
  			 \item les ensembles initiaux.
  			 \item le test d’appartenance de l’élément $c$ à $X$.
  			 \item le test d’appartenance de l’élément $a$ à $Y$.
  			 \item les ensembles $X - Y$ et $Y - X$.
  			 \item l’ensemble $X \cup Y$ (union).
  			 \item l'ensemble $X \cap Y$ (intersection).
	 \end{enumerate}
\end{exercise}

\begin{solution}
Il faut noter qu'il existe une solution qui se base sur le Python builtuints en utilisant la structure de donnée \textit{sets}. Mais comme en n'est dans la logique en utilise 
\begin{python}
from sympy import FiniteSet

X = FiniteSet('a', 'b', 'c', 'd')
Y = FiniteSet('s', 'b', 'd')

class MyClass(Yourclass):
    def __init__(self, my, yours):
        bla = '5 1 2 3 4'
        print bla
\end{python}
\begin{python}
class MyClass(Yourclass):
    def __init__(self, my, yours):
        bla = '5 1 2 3 4'
        print bla
\end{python}

\end{solution}
%%
\section{Premiers pas en Python symbolique}

This is an example of theorems.

\subsection{Variable et affection}\index{Variable et affection}

\begin{exercise}
Affectez les variables temps $t$ et distance $d$ par les valeurs 6.892 et 19.7. Calculez et affichez la valeur de la vitesse. Améliorez l’affichage en imposant un chiffre après le point décimal.
\end{exercise}

\begin{solution}
Pour, affectez des variables est les rendre symbolique comme c'est décrit dans le mémo ou il 
sera expliquer temps $t$ et distance $d$ par les valeurs 6.892 et 19.7. Calculez et affichez la 
valeur de la vitesse. Améliorez l’affichage en imposant un chiffre après le point décimal.
\end{solution}

\subsection{Contrôle du flux d’instructions}\index{Theorems!Single Line}
This is a theorem consisting of just one line.

\begin{exercise}
A set $\mathcal{D}(G)$ in dense in $L^2(G)$, $|\cdot|_0$. 
\end{exercise}
\begin{solution}
\end{solution}
%------------------------------------------------

\section{Les Fonctions}\index{Fonctions}

This is an example of a definition. A definition could be mathematical or it could define a concept.

\begin{exercise}
Écrire une fonction cube qui retourne le cube de son argument
\end{exercise}

\begin{exercise}
Écrire une fonction $volumeSphere$ qui calcule le volume d’une sphère de rayon $r$ fourni
en argument et qui utilise la fonction cube .
Tester la fonction $volumeSphere$ par un appel dans le programme principal.
\end{exercise}

\begin{exercise}
Écrire une fonction maFonction qui retourne $f(x) = 2x^{3} + x - 5$
\end{exercise}

\begin{exercise}
Écrire une fonction tabuler avec quatre paramètres : $fonction$ , $borneInf$ , $borneSup$
et $nbPas$ . Cette procédure affiche les valeurs de $fonction$ , de $borneInf$ à $borneSup$ ,
tous les nbPas . Elle doit respecter $borneInf < borneSup$.
Tester cette fonction par un appel dans le programme principal après avoir saisi les
deux bornes dans une floatbox et le nombre de pas dans une integerbox (utilisez le
module easyguiB ).
\end{exercise}

\begin{exercise}
Écrire une fonction $volMasse$ Ellipsoide qui retourne le volume et la masse d’un ellipsoïde grâce à un tuple. Les paramètres sont les trois demi-axes et la masse volumique. On donnera à ces quatre paramètres des valeurs par défaut. \\
On donne: $v = \frac{3}{4} \pi abc$ \\
Tester cette fonction par des appels avec différents nombres d’arguments.
\end{exercise}

\begin{exercise}
Une fonction $f (x)$ est lin\'eaire et a une valeur de $29$ \`a $x = -2$ et $39$ à $x = 3$. Trouver sa valeur à $x = 5$.
\end{exercise}

\begin{exercise}
Pour l'ensemble $N$ de nombres naturels et une opération binaire $f: N x N \longrightarrow N$, on appelle un élément $z$ $\epsilon$ $N$ une identité pour $f$, si $f (a, z) = a = f (z, a)$, pour tout a $\epsilon$ $N$. Lesquelles des opérations binaires suivantes ont une identité?:
\begin{enumerate}
  \item $f (x, y) = x + y - 3$
  \item $f (x, y) = max(x, y)$
  \item $f (x, y) = x^{y}$
\end{enumerate}
\end{exercise}
\begin{solution}
le deuxième et le troisième 
\end{solution}
\section{Structures de données}

\subsection{Logique}
\begin{exercise}
Dans la carte de Karnaugh ci-dessous, $X$ indique un terme sans intérêt. Quelle est la forme minimale de la fonction représentée par la carte de Karnaugh?
\end{exercise}
\subsection{Ensembles}
La notion d'objet immuable en Python est fondamentale,  une structure qui rappel les ensembles en mathématiques que soit fini ou infini est \textit{set}, importante, bien que
dans le cadre de SymPy elle s'appui entièrement sur Python avec certain modification, avec la collection d'objet.
\\

\textit{La fonction set accepte donc en argument un objet de type quelconque et s'efforce de le traduire dans un ensemble. Lorsqu'on ne passe aucun argument à set (option 2), ou qu'on lui passe une liste vide, set renvoie naturellement un ensemble vide; on aurait pu utiliser aussi bien, de la même manière, set(()), set({}), ou même set('') pour arriver au même résultat.}

	\begin{exercise}
		Définir deux ensembles $X = \lbrace a, b, c, d\rbrace$ et  $Y = \lbrace s, b, d\rbrace$ , puis 			affichez les résultats suivants :
 		\begin{enumerate}
  			 \item les ensembles initiaux.
  			 \item le test d’appartenance de l’élément $c$ à $X$.
  			 \item le test d’appartenance de l’élément $a$ à $Y$.
  			 \item les ensembles $X - Y$ et $Y - X$.
  			 \item l’ensemble $X \cup Y$ (union).
  			 \item l'ensemble $X \cap Y$ (intersection).
	 \end{enumerate}
	\end{exercise}

\begin{solution}
Il faut noter qu'il existe une solution qui se base sur le Python builtuints en utilisant la structure de donnée \textit{sets}. Mais comme en n'est dans la logique en utilise 
\begin{python}
from sympy import FiniteSet

X = FiniteSet('a', 'b', 'c', 'd')
Y = FiniteSet('s', 'b', 'd')

class MyClass(Yourclass):
    def __init__(self, my, yours):
        bla = '5 1 2 3 4'
        print bla
\end{python}
\begin{python}
class MyClass(Yourclass):
    def __init__(self, my, yours):
        bla = '5 1 2 3 4'
        print bla
\end{python}

\end{solution}
%\begin{exercise}
%Si $P$, $Q$, $R$ sont des sous-ensembles de l'ensemble universel $U$, alors $(P \cap Q\capR) \cup (P^{c} \cap Q \cap R)\cup(Q^{x}\cupR^{x})$ si
%\begin{enumerate}
%  \item  
%  \item 
%  \item
%\end{enumerate}
%\end{exercise}
%------------------------------------------------
\subsection{Polynômes}
\begin{exercise}
Considérons le polynôme $p(x) = a_{0} + a_{1} x + a_{2} x^{2} + a_{3} x^{3}$, où $a_{i} \neq 0$ $\forall i$. Le nombre minimum de multiplications nécessaires pour évaluer $p$ sur une entrée $x$ est:
\end{exercise}
%-----------------------------------------------
\subsection{Nombres parfaits et nombres chanceux}
\begin{exercise}
$\sqrt{12}$
\end{exercise}
%------------------------------------------------