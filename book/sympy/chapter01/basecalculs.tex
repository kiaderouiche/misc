\chapter{Premier pas vers SymPy}

Ce chapitre d’introduction présente la tournure d’esprit de la bibliothèque mathématique SymPy. Les 
autres chapitres de cette partie développent les notions de base de SymPy: effectuer des calculs 
numériques ou symboliques en analyse, opérer sur des vecteurs et des matrices, écrire des programmes, 
manipuler des listes de données, construire des graphiques, etc. Les parties suivantes de cet ouvrage 
approfondissent quelques branches des mathématiques dans lesquelles l’informatique fait preuve d’une 
grande efficacité.

\section{La bibliothèque SymPy}

\subsection{Le cas de la bibliothèque SymPy}

Dans un cas plus simple l'exemple 1.1 se formule beaucoup plus dans un outil comme SymPy est une bibliothèque de calcul formel elle est aussi un environnement pour 
l’apprentissage de l’algèbre, l’analyse, géométrie, combinatoire, cryptographie, mécanique 
classique et quantique pour le lycée et l’université mais aussi un environnement de 
développement et de recherche. SymPy  écrit entièrement en Python un langage de 
programmation facile à apprendre et adapté à l’apprentissage,  elle fourni aux étudiant 
\textit{SymPyGamma} une application web   notamment des primitives générales de traitement des 
expressions algébriques (développement, factorisation, …), des aides à l’organisation des objets 
mathématiques intervenant dans la résolution d’un problème ainsi qu’une assistance à la preuve. Il 
permet au professeur de préparer et de suivre le travail de l’élève. Différentes maquettes ont été 
développées et testées auprès d’élèves. Dans la plus récente, nous nous sommes attachés à explorer une 
nouvelle forme d’activité algébrique. Alors que le calcul en papier crayon et les logiciels standards 
considèrent  les expressions de façon isolée, l’environnement que nous développons organise en réseau 
les différentes expressions intervenant dans la résolution d’un problème. L’ordinateur peut facilement 
mettre à jour ce réseau quand l’utilisateur modifie certains de ses éléments. Il devient ainsi possible, 
pour aborder un problème générique, d’explorer facilement des cas particuliers et de conduire une 
généralisation. Les relations entre expressions algébriques sont mieux mises en évidence du fait de leur 
invariance dans les modifications du réseau. De façon très concise, Casyopée peut être défini
\subsubsection{SymPyGamma}
Est une interface onWeb marche avec un navigateur contient plusieurs catégorie liée de calcul, dynamique. L’Intérêt de cette outil qu'il est facilement partageable adapté pour l’enseignement et surtout l'auto-apprentissage

\includegraphics[scale=0.3]{../Pictures/sympyGammaMain.png} 

\subsubsection{SymPyLive}

\section{SymPy comme calculatrice}
\subsection{Premier calculs}
\subsubsection{Variables Python}
\subsubsection{Variables Symboliques}

\subsection{Structure de données dans SymPy}
\subsection{Variable et affection}\index{Variable et affection}

\begin{exercise}
Affectez les variables temps $t$ et distance $d$ par les valeurs 6.892 et 19.7. Calculez et affichez la valeur de la vitesse. Améliorez l’affichage en imposant un chiffre après le point décimal.
\end{exercise}

\begin{solution}
Pour, affectez des variables est les rendre symbolique comme c'est décrit dans le mémo ou il 
sera expliquer temps $t$ et distance $d$ par les valeurs 6.892 et 19.7. Calculez et affichez la 
valeur de la vitesse. Améliorez l’affichage en imposant un chiffre après le point décimal.
\end{solution}

\subsection{Contrôle du flux d’instructions}\index{Theorems!Single Line}
This is a theorem consisting of just one line.

\begin{exercise}
A set $\mathcal{D}(G)$ in dense in $L^2(G)$, $|\cdot|_0$. 
\end{exercise}
\begin{solution}
\end{solution}
%------------------------------------------------

\section{Les Fonctions}\index{Fonctions}

This is an example of a definition. A definition could be mathematical or it could define a concept.

\begin{exercise}
Écrire une fonction cube qui retourne le cube de son argument
\end{exercise}

\begin{exercise}
Écrire une fonction $volumeSphere$ qui calcule le volume d’une sphère de rayon $r$ fourni
en argument et qui utilise la fonction cube .
Tester la fonction $volumeSphere$ par un appel dans le programme principal.
\end{exercise}

\begin{exercise}
Écrire une fonction maFonction qui retourne $f(x) = 2x^{3} + x - 5$
\end{exercise}

\begin{exercise}
Écrire une fonction tabuler avec quatre paramètres : $fonction$ , $borneInf$ , $borneSup$
et $nbPas$ . Cette procédure affiche les valeurs de $fonction$ , de $borneInf$ à $borneSup$ ,
tous les nbPas . Elle doit respecter $borneInf < borneSup$.
Tester cette fonction par un appel dans le programme principal après avoir saisi les
deux bornes dans une floatbox et le nombre de pas dans une integerbox (utilisez le
module easyguiB ).
\end{exercise}

\begin{exercise}
Écrire une fonction $volMasse$ Ellipsoide qui retourne le volume et la masse d’un ellipsoïde grâce à un tuple. Les paramètres sont les trois demi-axes et la masse volumique. On donnera à ces quatre paramètres des valeurs par défaut. \\
On donne: $v = \frac{3}{4} \pi abc$ \\
Tester cette fonction par des appels avec différents nombres d’arguments.
\end{exercise}

\begin{exercise}
Une fonction $f (x)$ est lin\'eaire et a une valeur de $29$ \`a $x = -2$ et $39$ à $x = 3$. Trouver sa valeur à $x = 5$.
\end{exercise}

\begin{exercise}
Pour l'ensemble $N$ de nombres naturels et une opération binaire $f: N x N \longrightarrow N$, on appelle un élément $z$ $\epsilon$ $N$ une identité pour $f$, si $f (a, z) = a = f (z, a)$, pour tout a $\epsilon$ $N$. Lesquelles des opérations binaires suivantes ont une identité?:
\begin{enumerate}
  \item $f (x, y) = x + y - 3$
  \item $f (x, y) = max(x, y)$
  \item $f (x, y) = x^{y}$
\end{enumerate}
\end{exercise}
\begin{solution}
le deuxième et le troisième 
\end{solution}

