\chapter{Chaos}\index{Mouvement d'un pendule}
\section{Un mot sur la simulation}
\section{Mouvement d'un double pendule}
\section{Cinétique des gaz}
La théorie cinétique des gaz a pour objet d'expliquer le comportement macroscopique d'un gaz à partir des caractéristiques des mouvements des particules qui le composent. Elle permet notamment de donner une interprétation microscopique aux notions de :
\begin{itemize}
 \item température : c'est une mesure de l'agitation des particules, plus précisément de leur énergie cinétique
 \item pression : la pression exercée par un gaz sur une paroi résulte des chocs des particules sur cette dernière. Elle est liée à leur quantité de mouvement
\end{itemize}

\subsection{Gaz parfait}
Dans cette section on s'intéresse à la loi empirique $PV=NK_{B}T$
\subsection{Équation d'état de van der Waals}
L’équation d'état de Van Der Waals s'écrit sous la forme extensive suivante:
\begin{equation}
\left( P + \frac{am^{2}}{V^{2}} \right) \left(V-nb \right) = nRT
\end{equation}


