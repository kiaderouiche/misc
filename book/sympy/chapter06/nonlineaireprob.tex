
Les sujets de ce chapitre sont du néanmoins axées sur des questions ou l'approche mathématique et 
physique et demandé 

\section{Chaos}\index{Mouvement d'un pendule}
Prenons une pause dans l'apprentissage de nouvelles techniques et algorithmes informatiques
pour un peu, et passer du temps en utilisant ce que nous avons appris jusqu'à présent pour enquêter sur quelque chose d'intéressant. Nous allons commencer avec quelque chose de familier: le simple pendule.
\subsection{Pendule simple}
Le pendule simple figure
\subsection{Pendule à deux bras}
\subsection{Mouvements d’un robot}
Qu'est ce qu'il faut savoir quand en veut modélisé le comportement d'un robot?. Et bien la réponse est tout simplement des mathématiques

\section{M\'ecanique et information quantique}
\section{Le modèle $\phi^{4}$}
 \subsection{LES DIAGRAMMES DE FEYNMAN}
 
\section{Solution non linéaire d'équation algébrique}\index{Solving Nonlinear Algebraic Equations}

Qu'est ce que non-linéaire et qu'est ce que une \'equation alg\'ebrique

Une \'equation alg\'ebrique est un polyn\^ome de la forme $P(x)$

\begin{equation}
\exp(-x)\sin(x) = \cos(x)
\end{equation}
