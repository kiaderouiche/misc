%----------------------------------------------------------------------------------------
%	PACKAGES AND OTHER DOCUMENT CONFIGURATIONS
%----------------------------------------------------------------------------------------

\documentclass[11pt,fleqn]{book} % Default font size and left-justified equations

\usepackage[top=3cm,bottom=3cm,left=3.2cm,right=3.2cm,headsep=10pt,letterpaper]{geometry} % Page margins

\usepackage{xcolor} % Required for specifying colors by name
\definecolor{ocre}{RGB}{52,177,201} % Define the orange color used for highlighting throughout the book%

% Font Settings
\usepackage{avant} % Use the Avantgarde font for headings
%\usepackage{times} % Use the Times font for headings
\usepackage{mathptmx} % Use the Adobe Times Roman as the default text font together with math symbols from the Sym­bol, Chancery and Com­puter Modern fonts
\usepackage{microtype} % Slightly tweak font spacing for aesthetics
\usepackage[utf8]{inputenc} % Required for including letters with accents
\usepackage[T1]{fontenc} % Use 8-bit encoding that has 256 glyphs
\usepackage{amsthm}
%%%

% Bibliography
\usepackage[style=alphabetic,sorting=nyt,sortcites=true,autopunct=true,babel=hyphen,hyperref=true,abbreviate=false,backref=true,backend=biber]{biblatex}
\addbibresource{bibliography.bib} % BibTeX bibliography file
\defbibheading{bibempty}{}

\input{template/structure} % Insert the commands.tex file which contains the majority of the structure behind the template

%----------------------------------------------------------------------------------------
%	Definitions of new commands
%----------------------------------------------------------------------------------------

\def\R{\mathbb{R}}
\newcommand{\cvx}{convex}
\begin{document}

%----------------------------------------------------------------------------------------
%	TITLE PAGE
%----------------------------------------------------------------------------------------

\begingroup
\thispagestyle{empty}
\AddToShipoutPicture*{\put(0,0){\includegraphics[scale=1.25]{esahubble}}} % Image background
\centering
\vspace*{5cm}
\par\normalfont\fontsize{35}{35}\sffamily\selectfont
\textbf{SymPy par la pratique(DRAFT(wip))}\\
{\LARGE Exemple et exercice avancée}\par % Book title
\vspace*{1cm}
{\Huge K.I.A Derouiche}\par % Author name
\endgroup

%----------------------------------------------------------------------------------------
%	COPYRIGHT PAGE
%----------------------------------------------------------------------------------------

\newpage
~\vfill
\thispagestyle{empty}

%----------------------------------------------------------------------------------------
%	TABLE OF CONTENTS
%----------------------------------------------------------------------------------------

\chapterimage{head1.png} % Table of contents heading image

\pagestyle{empty} % No headers

\tableofcontents % Print the table of contents itself

%\cleardoublepage % Forces the first chapter to start on an odd page so it's on the right

\pagestyle{fancy} % Print headers again

%----------------------------------------------------------------------------------------
%	AVANT PROPOS
%----------------------------------------------------------------------------------------

\section{Avant-Propos}
Ce livre traite de Python, un langage de programmation de haut niveau, orienté objet,
totalement libre et terriblement efficace, conçu pour produire du code de qualité, 
portable et facile à intégrer. Ainsi la conception d'un programme Python est très rapide et
offre au développeur une bonne productivité. En tant que langage dynamique, il est
très souple d'utilisation et constitue un complément idéal à des langages compilés.
Il reste un langage complet et autosuffisant, pour des petits scripts fonctionnels de 
quelques lignes, comme pour des applicatifs complexes de plusieurs centaines de modules.

\subsection*{Pourquoi ce livre ?}
Il existe déjà de nombreux ouvrages excellents traduits de l'anglais qui traitent de
Python voire en présentent intégralité des modules disponibles. Citons Python en
concentré, le manuel de référence de Mark Lutz et David Ascher, aux éditions
O'Reilly, ou encore Apprendre à programmer avec Python de Gérard Swinnen, aux
éditions Eyrolles, inspiré en partie du texte How to think like a computer scientist
(Downey, Elkner, Meyers), et comme son titre l'indique, tr\'es p\'edadogique.
Alors, pourquoi ce livre ?

\subsection*{A qui s'adresse l'ouvrage?}

\chapterimage{head2.png} % Chapter heading image
\section{Calcul formel}
Le calcul formel, ou parfois calcul symbolique, est le domaine des mathématiques et de informatique qui s'intéresse aux algorithmes opérant sur des objets de nature mathématique par le biais de représentations finies et exactes. Ainsi, un nombre entier est représenté de manière finie et exacte par la suite des chiffres de son écriture en base 2. Étant données les représentations de deux nombres entiers, le calcul formel se pose par exemple la question de calculer celle de leur produit.

Le calcul formel est en général considéré comme un domaine distinct du calcul scientifique, cette dernière appellation faisant référence au calcul numérique approché à l'aide de nombres en virgule flottante, là où le calcul formel met l'accent sur les calculs exacts sur des expressions pouvant contenir des variables ou des nombres en précision arbitraire (en). Comme exemples d'opérations de calcul formel, on peut citer le calcul de dérivées ou de primitives, la simplification d'expressions, la décomposition en facteurs irréductibles de polynômes, la mise sous formes normales de matrices, ou encore la résolution des systèmes polynomiaux.

Sur le plan théorique, on s'attache en calcul formel à donner des algorithmes avec la démonstration qu'ils terminent en temps fini et la démonstration que le résultat est bien la représentation d'un objet mathématique défini préalablement. Autant que possible, on essaie de plus d'estimer la complexité des algorithmes que l'on décrit, c'est-à-dire le nombre total d'opérations élémentaires qu'ils effectuent. Cela permet d'avoir une idée a priori du temps d'exécution d'un algorithme, de comparer l'efficacité théorique de différents algorithmes ou encore éclairer la nature même du problème.
\subsection{Logiciel de système de calcul formel}
Dans cette section en va exposer les systèmes de calcul formel, leur intérêt qui à vue un renouveau ces dernières années à cause de l'émergence de technique, technologie et nouvelle approche de programmation pour le domaine scientifique et industriel, hormis le fait que le logiciel de calcul formel en soient sont un outil pédagogique 
indispensable pour les scientifiques et les ingénieurs

\begin{definition}
Un logiciel de système formel est un outil qui facilite le calcul symbolique. La partie principale de ce système est la manipulation des expressions mathématiques sous leur forme symbolique.
\end{definition}

\begin{example}
soit $G=\{x\in\mathbb{R}^2:|x|<3\}$ et noté par: $x^0=(1,1)$; en considère la fonction:
\begin{equation}
f(x)=\left\{\begin{aligned} & \mathrm{e}^{|x|} & & \text{si $|x-x^0|\leq 1/2$}\\
& 0 & & \text{si $|x-x^0|> 1/2$}\end{aligned}\right.
\end{equation}
The function $f$ has bounded support, we can take $A=\{x\in\mathbb{R}^2:|x-x^0|\leq 1/2+\epsilon\}$ for all $\epsilon\in\intoo{0}{5/2-\sqrt{2}}$.
\end{example}

cet exemple se traduit en forme symbolique avec la bibliothèque SymPy:

\subsection{Quelques logiciels de calcul formel}

qui exprime ce qui nous permet notre choix pour un CAS qui possède des caractéristiques techniques et sur le plan du coût très important quand peut résumer dans les points suivants:
\begin{enumerate}
	\item Leger. 
	\item S’appuie sur le langage de programmation Python.
	\item Portabilité dans toute transparence.
\end{enumerate}

L'un des systèmes qui peut nous permettre d'écrire cette exemple avec un ordinateurs avec SymPy qui semble mieux intégré
\subsection{SymPy vs Sagemath}
Peut être que parmi les CAS les plus proche de SymPy est sans aucun doute Sagemath[]
\subsection{Pourquoi choisir SymPy?}
\section{Passer du problème au codage du problème}
C'est simple, simplifier s'il est question d'une expression algébrique puis passé au code SymPy!!!

%----------------------------------------------------------------------------------------
%	CHAPTER 1
%----------------------------------------------------------------------------------------
\chapter{Premier pas vers SymPy}

Ce chapitre d’introduction présente la tournure d’esprit de la bibliothèque mathématique SymPy. Les 
autres chapitres de cette partie développent les notions de base de SymPy: effectuer des calculs 
numériques ou symboliques en analyse, opérer sur des vecteurs et des matrices, écrire des programmes, 
manipuler des listes de données, construire des graphiques, etc. Les parties suivantes de cet ouvrage 
approfondissent quelques branches des mathématiques dans lesquelles l’informatique fait preuve d’une 
grande efficacité.

\section{La bibliothèque SymPy}

\subsection{Le cas de la bibliothèque SymPy}

Dans un cas plus simple l'exemple 1.1 se formule beaucoup plus dans un outil comme SymPy est une bibliothèque de calcul formel elle est aussi un environnement pour 
l’apprentissage de l’algèbre, l’analyse, géométrie, combinatoire, cryptographie, mécanique 
classique et quantique pour le lycée et l’université mais aussi un environnement de 
développement et de recherche. SymPy  écrit entièrement en Python un langage de 
programmation facile à apprendre et adapté à l’apprentissage,  elle fourni aux étudiant 
\textit{SymPyGamma} une application web   notamment des primitives générales de traitement des 
expressions algébriques (développement, factorisation, …), des aides à l’organisation des objets 
mathématiques intervenant dans la résolution d’un problème ainsi qu’une assistance à la preuve. Il 
permet au professeur de préparer et de suivre le travail de l’élève. Différentes maquettes ont été 
développées et testées auprès d’élèves. Dans la plus récente, nous nous sommes attachés à explorer une 
nouvelle forme d’activité algébrique. Alors que le calcul en papier crayon et les logiciels standards 
considèrent  les expressions de façon isolée, l’environnement que nous développons organise en réseau 
les différentes expressions intervenant dans la résolution d’un problème. L’ordinateur peut facilement 
mettre à jour ce réseau quand l’utilisateur modifie certains de ses éléments. Il devient ainsi possible, 
pour aborder un problème générique, d’explorer facilement des cas particuliers et de conduire une 
généralisation. Les relations entre expressions algébriques sont mieux mises en évidence du fait de leur 
invariance dans les modifications du réseau. De façon très concise, Casyopée peut être défini
\subsection{Travaillez avec SymPy}
Pour utiliser Sage, il suffit d’un navigateur web tel que Firefox. Le plus simple dans un premier temps est de se connecter sur un serveur de calcul Sage public comme http://sagenb.org/. On a alors accès à une interface utilisateur riche constituée d’un « bloc-notes » (notebook en anglais) permettant d’éditer
et partager des feuilles de travail (worksheets, voir figure 1.2). De nombreuses universités et institutions mettent de tels serveurs à la disposition de leurs utilisateurs ; renseignez-vous autour de vous. Pour un usage plus intensif, on

installe généralement Sage sur sa propre machine 1 . On peut alors toujours
l’utiliser avec l’interface bloc-notes dans le navigateur web. Alternativement, on
peut l’utiliser en ligne de commande, comme une calculatrice (voir figure 1.3).
Les figures 1.4 et 1.5 illustrent pas à pas les étapes pour accéder à un serveur
Sage public et l’utiliser pour faire un calcul :


\subsubsection{SymPyGamma}
Est une interface onWeb marche avec un navigateur contient plusieurs catégorie liée de calcul, dynamique. L’Intérêt de cette outil qu'il est facilement partageable adapté pour l’enseignement et surtout l'auto-apprentissage

\includegraphics[scale=0.3]{../Pictures/sympyGammaMain.png} 

\subsubsection{SymPyLive}
SymPy Live est SymPy qui s'exécute sur Google App Engine. Ceci est juste un shell Python standard, avec les commandes suivantes exécutées par défaut
\section{SymPy comme calculatrice}
Contrairement à Sage[], Maple, Octave et les autres logiciel de calcul formel, SymPy, effectue des calculs directe numériques de deux manière différents en passant par la méthode, 
\subsection{Premier calculs}
\subsubsection{Variables Python}
Lorsque l’on veut conserver le résultat d’un calcul, on peut l’affecter à une
variable :

\subsubsection{Variables Symboliques}
Les objets mathématiques manipulés par SymPy sont symboliques ils sont représentés exactement loin de 
toute approximation numérique, SymPy permet une manipulation avec des expressions contenant
des variables, comme $x^{2} + zy^{3} + z^{2}$ ou encore $sin(x) - exp(x)$. Les variables symboliques
du mathématicien $x$, $y$, $z$ apparaissant dans ces expressions diffèrent, avec SymPy,
des variables du développeur $sin(2) = 0.9092974268256817$ que nous manipulons sous Python 
section précédente. 


\subsection{Structure de données dans SymPy}
Le moteur symbolique de SymPy tire parti de l'orientation des objets (notamment l'héritage) pour créer une base de code facilement extensible. Toutes les classes dérivent des fonctionnalités, telles que la possibilité de se comparer à d'autres objets, à partir de méthodes de la super-classe Basic. Les objets pouvant faire l'objet d'opérations algébriques acquièrent cette capacité grâce à un ensemble de méthodes d'une classe appelée Expr. Ces objets Expr peuvent être conservés dans des objets conteneur (qui contiennent également la sous-classe Expr) Mul, Add et Pow; les objets conteneur sont instanciés à l'aide de l'opérateur Python, tel que la fonction de surcharge, qui permet au constructeur de la classe conteneur d'être appelé chaque fois que l'opérateur binaire approprié est utilisé (* pour Mul, + pour Ajouter et $**$ pour Pow).
De cette manière, des objets supplémentaires peuvent être ajoutés en créant simplement une sous-classe qui hérite des fonctionnalités de la classe Expr. Ces sous-classes bénéficient gratuitement de certaines fonctionnalités, telles que la possibilité de comparer, de multiplier, d’ajouter, etc. Voici comment SymPy crée un environnement modifiable, maintenable, et donc facile à étendre. Grâce à la possibilité d'hériter des propriétés de classes supérieures, la quantité de code nécessaire pour développer, par exemple, un système modélisant la mécanique quantique et la notation Dirac décroissant de manière significative.

\includegraphics[scale=0.3]{../Pictures/sympyarch.png} 
\subsection{Variable et affection}\index{Variable et affection}

\begin{exercise}
Affectez les variables temps $t$ et distance $d$ par les valeurs 6.892 et 19.7. Calculez et affichez la valeur de la vitesse. Améliorez l'affichage en imposant un chiffre après le point décimal.
\end{exercise}

\begin{solution}
Pour, affectez des variables est les rendre symbolique comme c'est décrit dans le mémo ou il 
sera expliquer temps $t$ et distance $d$ par les valeurs 6.892 et 19.7. Calculez et affichez la 
valeur de la vitesse. Améliorez l'affichage en imposant un chiffre après le point décimal.
\end{solution}

\subsection{Contrôle du flux d’instructions}\index{Theorems!Single Line}
This is a theorem consisting of just one line.

\begin{exercise}
A set $\mathcal{D}(G)$ in dense in $L^2(G)$, $|\cdot|_0$. 
\end{exercise}
\begin{solution}
\end{solution}
%------------------------------------------------

\section{Les Fonctions}\index{Fonctions}

This is an example of a definition. A definition could be mathematical or it could define a concept.

\begin{exercise}
Écrire une fonction cube qui retourne le cube de son argument
\end{exercise}

\begin{exercise}
Écrire une fonction $volumeSphere$ qui calcule le volume d’une sphère de rayon $r$ fourni
en argument et qui utilise la fonction cube .
Tester la fonction $volumeSphere$ par un appel dans le programme principal.
\end{exercise}

\begin{exercise}
Écrire une fonction maFonction qui retourne $f(x) = 2x^{3} + x - 5$
\end{exercise}

\begin{exercise}
Écrire une fonction tabuler avec quatre paramètres : $fonction$ , $borneInf$ , $borneSup$
et $nbPas$ . Cette procédure affiche les valeurs de $fonction$ , de $borneInf$ à $borneSup$ ,
tous les nbPas . Elle doit respecter $borneInf < borneSup$.
Tester cette fonction par un appel dans le programme principal après avoir saisi les
deux bornes dans une floatbox et le nombre de pas dans une integerbox (utilisez le
module easyguiB ).
\end{exercise}

\begin{exercise}
Écrire une fonction $volMasse$ Ellipsoide qui retourne le volume et la masse d’un ellipsoïde grâce à un tuple. Les paramètres sont les trois demi-axes et la masse volumique. On donnera à ces quatre paramètres des valeurs par défaut. \\
On donne: $v = \frac{3}{4} \pi abc$ \\
Tester cette fonction par des appels avec différents nombres d’arguments.
\end{exercise}

\begin{exercise}
Une fonction $f (x)$ est lin\'eaire et a une valeur de $29$ \`a $x = -2$ et $39$ à $x = 3$. Trouver sa valeur à $x = 5$.
\end{exercise}

\begin{exercise}
Pour l'ensemble $N$ de nombres naturels et une opération binaire $f: N x N \longrightarrow N$, on appelle un élément $z$ $\epsilon$ $N$ une identité pour $f$, si $f (a, z) = a = f (z, a)$, pour tout a $\epsilon$ $N$. Lesquelles des opérations binaires suivantes ont une identité?:
\begin{enumerate}
  \item $f (x, y) = x + y - 3$
  \item $f (x, y) = max(x, y)$
  \item $f (x, y) = x^{y}$
\end{enumerate}
\end{exercise}
\begin{solution}
le deuxième et le troisième 
\end{solution}


%----------------------------------------------------------------------------------------
%	CHAPTER 2
%----------------------------------------------------------------------------------------



\section{Programmation Orientée Objet}\index{Notations}
\begin{notation}
Given an open subset $G$ of $\mathbb{R}^n$, the set of functions $\varphi$ are:
\begin{enumerate}
\item Bounded support $G$;
\item Infinitely differentiable;
\end{enumerate}
a vector space is denoted by $\mathcal{D}(G)$. 
\end{notation}

 \subsection{POO}
\begin{exercise}
 Définir une classe Vecteur2D avec un constructeur fournissant les coordonnées par
défaut d’un vecteur du plan (par exemple : $x = 0$ et $y = 0$ ).
Dans le programme principal, instanciez un Vecteur2D sans paramètre, un Vecteur2D
avec ses deux paramètres, et affichez-les.
\end{exercise}
\begin{solution}
 en utilise le module sympy.geometry ce module fait appel à tout les outils et theories qui
 peuvents entre utiliser dans le cade de la géométrie dans le Plan.
 \begin{python}
 from sympy.geometry
  \end{python}
\end{solution}

\begin{exercise}
Enrichissez la classe Vecteur2D précédente en lui ajoutant une méthode d’affichage
et une méthode de surcharge d’addition de deux vecteurs du plan.
Dans le programme principal, instanciez deux Vecteur2D , affichez-les et affichez leur
somme.
\end{exercise}
\begin{solution}
\end{solution}

%------------------------------------------------
\subsection{Notions de COO et d’encapsulation}
%------------------------------------------------

%%
%\includeonly{geometry/euclid}


%---------------------------------------------------------------------------------------
%	CHAPTER 3
%----------------------------------------------------------------------------------------
\chapter{Problème non linéaire}

Les sujets de ce chapitre sont du néanmoins axées sur des questions ou l'approche mathématique et 
physique et demandé 

\section{Chaos}\index{Mouvement d'un pendule}
Prenons une pause dans l'apprentissage de nouvelles techniques et algorithmes informatiques
pour un peu, et passer du temps en utilisant ce que nous avons appris jusqu'à présent pour enquêter sur quelque chose d'intéressant. Nous allons commencer avec quelque chose de familier: le simple pendule.
\subsection{Pendule simple}
Le pendule simple figure
\subsection{Pendule à deux bras}
\subsection{Mouvements d’un robot}
Qu'est ce qu'il faut savoir quand en veut modélisé le comportement d'un robot?. Et bien la réponse est tout simplement des mathématiques

\section{M\'ecanique et information quantique}
\section{Le modèle $\phi^{4}$}
 \subsection{LES DIAGRAMMES DE FEYNMAN}
 
\section{Solution non linéaire d'équation algébrique}\index{Solving Nonlinear Algebraic Equations}

Qu'est ce que non-linéaire et qu'est ce que une \'equation alg\'ebrique

Une \'equation alg\'ebrique est un polyn\^ome de la forme $P(x)$

\begin{equation}
\exp(-x)\sin(x) = \cos(x)
\end{equation}

%-------------------------------------------------------------------------------------------------
% CHAPTER 4
%-------------------------------------------------------------------------------------------------
\chapter{Mathématique pures}
\section{La théorie de catégorie}
L’Intérêt de la théorie de catégorie dans les mathématiques modernes et l'informatique théorique et sui dépasse de loin, cette derniére pour \^etre
\section{Transport optimal}
C'est quoi le \textit{ transport optimal}?, exemple simple..., le domaine du transport optimal
est très 

%-------------------------------------------------

\section{Figure}\index{Figure}

\begin{table}[h]
\centering
\begin{tabular}{l l l}
\toprule
\textbf{Treatments} & \textbf{Response 1} & \textbf{Response 2}\\
\midrule
Treatment 1 & 0.0003262 & 0.562 \\
Treatment 2 & 0.0015681 & 0.910 \\
Treatment 3 & 0.0009271 & 0.296 \\
\bottomrule
\end{tabular}
\caption{Table caption}
\end{table}

%
%\begin{figure}[h]
%\centering\includegraphics[scale=0.5]
%\caption{Figure caption}
%\end{figure}

%----------------------------------------------------------------------------------------
%	Technique Avancée
%----------------------------------------------------------------------------------------

\chapter{Annexe}

\section{Programmation Orientée Objet}\index{Notations}
\section{D\'ecorateurs}
Les d\'ecorateurs un m\'ecanisme incontournable pour \'ecrire de tr\'es bon code et purement 
lisible et portable
\subsection{Optimisation du code}
Sans aucun doute l'usage de la programmation symbolique avec ce que en a vue plus haut, ralentisse 
grandement l'exécution du programme, donc en gagne sur le coté sureté, élégance
et maintenance du code et d'autre part en perd complètement la vitesse; penser à des centaine de ligne 
de code si vous voulez programmé un robot, voiture ou des objets connectés qui implémente des 
algorithmes mathématiques et qui de demande beaucoup de ressource est un temps de retour très élevées 

\subsection{Cython}
Cython (http://www.cython.org/ ) est un métalangage qui permet de combiner du code
Python et des types de donn\'ees C, pour concevoir des extensions compilables pour
Python.
Dans un module Cython, il est possible de définir des variables C directement dans
le code Python et de définir des fonctions C qui prennent en paramètre des
variables C ou des objets Python.
Cython contr\^ole ensuite de manière transparente la génération de l’extension C, en
transformant le module en code C par le biais des API C de Python.
Toutes les fonctions Python du module sont alors automatiquement publiées.
Le gain de temps dans la conception introduit par Cython est considérable : toute la
mécanique habituellement mise en œuvre pour créer un module d’extension est
entièrement gérée par Cython.
Ainsi, la fonction max() du module calculs.c pr\'ec\'edemment présent\'ee devient :

Les fichiers Cython ont par convention l’extension pyx, en référence à l’ancien nom.

setup.py pour calculs.pyx

\begin{python}
from distutils.core import setup
from distutils.extension import Extension
from Cython.Distutils import build_ext

extension = Extension("calculs", ["calculs.pyx"])

setup(name="calculs", ext_modules=[extension],cmdclass={'build_ext': build_ext})

\end{python}

\subsection{Theano}
Theano est une bibliothèque pour l'accélération du code lent en Python, tr\'es importante et intéressante
elle offre une syntaxe très particulière.
\section{Interface graphique}
Quelles bibliothèque Python pour développé des applications scientifiques graphiques, le choix est 
difficile. D'autant qu'il y en a plusieurs pour ne cité que les plus populaires: Tkinter, Gtk, Qt, wx  
il existe encore d'autre bibliothèques qui sont moins con\c{c}u: Ftk 

Dans cette section nous allons exposés les bibliothèques les plus populaires en mettent l'accent plus particulièrement sur deux d'entre eux: Qt et ipywdigets. 

\section{Bibliographie}
\section{Index}

\end{document}