%%%%%%%%%%%%%%%%%%%%%%%%%%%%%%%%%%%%%%%%%
%  My documentation report
%  Objetive: Explain what I did and how, so someone can continue with the investigation
%
% Important note:
% Chapter heading images should have a 2:1 width:height ratio,
% e.g. 920px width and 460px height.
%
%%%%%%%%%%%%%%%%%%%%%%%%%%%%%%%%%%%%%%%%%


%----------------------------------------------------------------------------------------
%	PACKAGES AND OTHER DOCUMENT CONFIGURATIONS
%----------------------------------------------------------------------------------------

\documentclass[11pt,fleqn]{book} % Default font size and left-justified equations

\usepackage[top=3cm,bottom=3cm,left=3.2cm,right=3.2cm,headsep=10pt,letterpaper]{geometry} % Page margins

\usepackage{xcolor} % Required for specifying colors by name
\definecolor{ocre}{RGB}{52,177,201} % Define the orange color used for highlighting throughout the book%

% Font Settings
\usepackage{avant} % Use the Avantgarde font for headings
%\usepackage{times} % Use the Times font for headings
\usepackage{mathptmx} % Use the Adobe Times Roman as the default text font together with math symbols from the Sym­bol, Chancery and Com­puter Modern fonts
\usepackage{microtype} % Slightly tweak font spacing for aesthetics
\usepackage[utf8]{inputenc} % Required for including letters with accents
\usepackage[T1]{fontenc} % Use 8-bit encoding that has 256 glyphs
\usepackage{amsthm}
%%%

% Bibliography
\usepackage[style=alphabetic,sorting=nyt,sortcites=true,autopunct=true,babel=hyphen,hyperref=true,abbreviate=false,backref=true,backend=biber]{biblatex}
\addbibresource{bibliography.bib} % BibTeX bibliography file
\defbibheading{bibempty}{}

\input{structure} % Insert the commands.tex file which contains the majority of the structure behind the template

%----------------------------------------------------------------------------------------
%	Definitions of new commands
%----------------------------------------------------------------------------------------

\def\R{\mathbb{R}}
\newcommand{\cvx}{convex}
\begin{document}

%----------------------------------------------------------------------------------------
%	TITLE PAGE
%----------------------------------------------------------------------------------------

\begingroup
\thispagestyle{empty}
\AddToShipoutPicture*{\put(0,0){\includegraphics[scale=1.25]{esahubble}}} % Image background
\centering
\vspace*{5cm}
\par\normalfont\fontsize{35}{35}\sffamily\selectfont
\textbf{SymPy par la pratique}\\
{\LARGE Exemple et exercice}\par % Book title
\vspace*{1cm}
{\Huge K.I.A Derouiche}\par % Author name
\endgroup

%----------------------------------------------------------------------------------------
%	COPYRIGHT PAGE
%----------------------------------------------------------------------------------------

\newpage
~\vfill
\thispagestyle{empty}

%\noindent Copyright \copyright\ 2014 Andrea Hidalgo\\ % Copyright notice

\noindent \textsc{Summer Research Internship, University of Western Ontario}\\

\noindent \textsc{github.com/LaurethTeX/Clustering}\\ % URL

\noindent This research was done under the supervision of Dr. Pauline Barmby with the financial support of the MITACS Globalink Research Internship Award within a total of 12 weeks, from June 16th to September 5th of 2014.\\ % License information

\noindent \textit{First release, August 2014} % Printing/edition date

%----------------------------------------------------------------------------------------
%	TABLE OF CONTENTS
%----------------------------------------------------------------------------------------

\chapterimage{head1.png} % Table of contents heading image

\pagestyle{empty} % No headers

\tableofcontents % Print the table of contents itself

%\cleardoublepage % Forces the first chapter to start on an odd page so it's on the right

\pagestyle{fancy} % Print headers again

%----------------------------------------------------------------------------------------
%	AVANT PROPOS
%----------------------------------------------------------------------------------------

\section{Avant-Propos}
Ce livre traite de Python, un langage de programmation de haut niveau, orienté objet,
totalement libre et terriblement efficace, conçu pour produire du code de qualité, 
portable et facile à intégrer. Ainsi la conception d'un programme Python est très rapide et
offre au développeur une bonne productivité. En tant que langage dynamique, il est
très souple d'utilisation et constitue un complément idéal à des langages compilés.
Il reste un langage complet et autosuffisant, pour des petits scripts fonctionnels de 
quelques lignes, comme pour des applicatifs complexes de plusieurs centaines de modules.

\subsection*{Pourquoi ce livre ?}
Il existe déjà de nombreux ouvrages excellents traduits de l'anglais qui traitent de
Python voire en présentent intégralité des modules disponibles. Citons Python en
concentré, le manuel de référence de Mark Lutz et David Ascher, aux éditions
O'Reilly, ou encore Apprendre à programmer avec Python de Gérard Swinnen, aux
éditions Eyrolles, inspiré en partie du texte How to think like a computer scientist
(Downey, Elkner, Meyers), et comme son titre l'indique, tr\'es p\'edadogique.
Alors, pourquoi ce livre ?

\subsection*{A qui s'adresse l'ouvrage?}

%----------------------------------------------------------------------------------------
%	CHAPTER 1
%----------------------------------------------------------------------------------------

\chapterimage{head2.png} % Chapter heading image
\section{Introduction}
Ce recueil d'exercices et de problèmes de programmation s'adresse aussi bien aux débutants qu'aux programmeurs confirmés. Il présente en effet plusieurs états d'esprit dont les deux principaux sont la programmation classique en Pascal pour les étudiants du premier cycle universitaire, et la programmation fonctionnelle en Lisp pour le second cycle.

Ce livre constitue un panorama (non exhaustif, mais suffisant) sur les langages de programmation, et offre une grande variété dans les sujets traités : graphiques, calcul matriciel, traitements de chaînes de caractères, graphes, intelligence artificielle...
\\
La première partie du livre sera consacré \`a la résolution par une approche symbolique au divers questions posées au étudiants et toute personnes qui aiment savoir et voir s'initier  pour des niveaux et des questions rencontrés, la deuxième partie du livre sera questions aux problèmes plus rencontrés pour des étudiants passionnée des questions entre mathématiques et technologies, chercheurs et développeurs d'applications scientifiques, la troisième partie plus consacré aux questions poussées  


\subsection{Pourquoi programmer en symbolique }


%----------------------------------------------------------------------------------------
%	CHAPTER 1
%----------------------------------------------------------------------------------------


\chapter{Calcul formel}

L'approche La simulation numérique est devenue essentielle dans de nombreux domaines tels que la mécanique des fluides et des solides, la météo, l'évolution du climat, la biologie ou les semi-conducteurs. Elle permet de comprendre, de prévoir, d'accéder là où les instruments de mesures s'arrêtent. 

Ce livre présente des méthodes performantes du calcul scientifique : matrices creuses, résolution efficace des grands systèmes linéaires, ainsi que de nombreuses applications à la résolution par éléments finis et différences finies. Alternant algorithmes et applications, les programmes sont directement présentés en langage C++. Ils sont sous forme concise et claire, et utilisent largement les notions de classe et de généricité du langage C++. 

Le contenu de ce livre a fait l'objet de cours de troisième année à l'école nationale supérieure d'informatique et de mathématiques appliquées de Grenoble (ENSIMAG) ainsi qu'au mastère de mathématiques appliquées de l'université Joseph Fourier. Des connaissances de base d'algèbre matricielle et de programmation sont recommandées. La maîtrise du contenu de cet ouvrage permet d'appréhender les principaux paradigmes de programmation du calcul scientifique. Il est alors possible d'appliquer ces paradigmes pour aborder des problèmes d'intérêt pratique, tels que la résolution des équations aux dérivées partielles, qui est abordée au cours de ce livre. La diversité des sujets abordés, l'efficacité des algorithmes présentés et leur écriture directe en langage C++ font de cet ouvrage un recueil fort utile dans la vie professionnelle d'un ingénieur. 

Le premier chapitre présente les bases fondamentales pour la suite : présentation du langage C++ à travers la conception d'une classe de quaternions et outils d'analyse asymptotique du temps de calcul des algorithmes. Le second chapitre aborde l'algorithme de transformée de Fourier rapide et développe deux applications à la discrétisation d'équations aux dérivées partielles par la méthode des différences finies. Le troisième chapitre est dédié aux matrices creuses et à l'algorithme du gradient conjugué. Ces notions sont appliquées à la méthode des éléments finis. En annexe sont groupés des exemples de génération de maillage et de visualisation graphique. 

S'il est cependant recommandé de maîtriser les notions du premier chapitre pour aborder le reste du livre, les chapitres deux et trois sont complètement indépendants et peuvent être abordés séparément. Ces chapitres sont complétés par des exercices qui en constituent des développements, ainsi que des notes bibliographiques retraçant l'historique des travaux et fournissant des références sur des logiciels et librairies récents implémentant ou étendant les algorithmes présentés. 

\section{Système de calcul formel}
Dans cette section en va exposer les systèmes de calcul formel, leur 
intérêt qui à vue un renouveau ces dernières années à cause de 
l'émergence de technique, technologie et nouvelle approche de 
programmation pour le domaine scientifique et industriel, hormis le fait 
que le logiciel de calcul formel en soient sont un outil pédagogique 
indispensable pour les scientifiques et les ingénieurs

\begin{example}
Let $G=\{x\in\mathbb{R}^2:|x|<3\}$ and denoted by: $x^0=(1,1)$; consider the function:
\begin{equation}
f(x)=\left\{\begin{aligned} & \mathrm{e}^{|x|} & & \text{si $|x-x^0|\leq 1/2$}\\
& 0 & & \text{si $|x-x^0|> 1/2$}\end{aligned}\right.
\end{equation}
The function $f$ has bounded support, we can take $A=\{x\in\mathbb{R}^2:|x-x^0|\leq 1/2+\epsilon\}$ for all $\epsilon\in\intoo{0}{5/2-\sqrt{2}}$.
\end{example}

qui exprime ce qui nous permet notre choix pour un CAS qui possède des caractéristiques techniques et sur le plan du coût très important quand peut résumer dans les points suivants:
\begin{enumerate}
	\item Leger et 
	\item S’appuie sur le langage de programmation Python
	\item Portabilité dans toute transparence
\end{enumerate}

L'un des systèmes qui peut nous permettre d'écrire cette exemple avec un ordinateurs avec SymPy qui semble mieu intégré

\section{Bibliothèque SymPy}

Dans un cas plus simple l'exemple 1.1 se formule beaucoup plus dans un outil comme SymPy est une bibliothèque de calcul formel elle est aussi un environnement pour 
l’apprentissage de l’algèbre, l’analyse, géométrie, combinatoire, cryptographie, mécanique 
classique et quantique pour le lycée et l’université mais aussi un environnement de 
développement et de recherche. SymPy  écrit entièrement en Python un langage de 
programmation facile à apprendre et adapté à l’apprentissage,  elle fourni aux étudiant 
\textit{SymPyGamma} une application web   notamment des primitives générales de traitement des expressions algébriques (développement, factorisation, …), des aides à l’organisation des objets mathématiques intervenant dans la résolution d’un problème ainsi qu’une assistance à la preuve. Il permet au professeur de préparer et de suivre le travail de l’élève. Différentes maquettes ont été développées et testées auprès d’élèves. Dans la plus récente, nous nous sommes attachés à explorer une nouvelle forme d’activité algébrique. Alors que le calcul en papier crayon et les logiciels standards considèrent
 les expressions de façon isolée, l’environnement que nous développons organise en réseau les différentes expressions intervenant dans la résolution d’un problème. L’ordinateur peut facilement mettre à jour ce réseau quand l’utilisateur modifie certains de ses éléments. Il devient ainsi possible, pour aborder un problème générique, d’explorer facilement des cas particuliers et de conduire une généralisation. Les relations entre expressions algébriques sont mieux mises en évidence du fait de leur invariance dans les modifications du réseau. De façon très concise, Casyopée peut être défini
\subsection{SymPyGamma}
Est une interface onWeb marche avec un navigateur contient plusieurs catégorie liée de calcul, dynamique. L'interet de cette outil qu'il est facilement partageable adapté pour l’enseignement et surtout l'auto-apprentissage 
\subsection{Besoin de rester dans le symbolique}
Le symbolique est une grande importance d'un point de vue technique, car il permet
de limité les risques de bug dans l'exécution des programmes, dans le contexte de 
la vérification formelle si en prend le programme suivant:

\subsection{Passage du symbolique au numérique}
Généralement, le symbolique parmis c'est  
\subsection{Faire des dessins}

%--------------------------------------------------------------------------------------
% CHAPTRE
%--------------------------------------------------------------------------------------

\chapter{Ensemble}

\section{Ensemble}

\subsection{Ensembles}
La notion d'objet immuable en Python est fondamentale,  une structure qui rappel les ensembles en mathématiques que soit fini ou infini est \textit{set}, importante, bien que
dans le cadre de SymPy elle s'appui entièrement sur Python avec certain modification, avec la collection d'objet.
\\

\textit{La fonction set accepte donc en argument un objet de type quelconque et s'efforce de le traduire dans un ensemble. Lorsqu'on ne passe aucun argument à set (option 2), ou qu'on lui passe une liste vide, set renvoie naturellement un ensemble vide; on aurait pu utiliser aussi bien, de la même manière, set(()), set({}), ou même set('') pour arriver au même résultat.}

%\begin{exercise}
%		Définir deux ensembles $X = \lbrace a, b, c, d\rbrace$ et  $Y = \lbrace s, b, d\rbrace$ , puis 			affichez les résultats suivants :
% 		\begin{enumerate}
%  			 \item les ensembles initiaux.
%  			 \item le test d’appartenance de l’élément $c$ à $X$.
%  			 \item le test d’appartenance de l’élément $a$ à $Y$.
%  			 \item les ensembles $X - Y$ et $Y - X$.
%  			 \item l’ensemble $X \cup Y$ (union).
%  			 \item l'ensemble $X \cap Y$ (intersection).
%	 \end{enumerate}
%\end{exercise}

\begin{solution}
Il faut noter qu'il existe une solution qui se base sur le Python builtuints en utilisant la structure de donnée \textit{sets}. Mais comme en n'est dans la logique en utilise 
\begin{python}
from sympy import FiniteSet

X = FiniteSet('a', 'b', 'c', 'd')
Y = FiniteSet('s', 'b', 'd')

class MyClass(Yourclass):
    def __init__(self, my, yours):
        bla = '5 1 2 3 4'
        print bla
\end{python}
\begin{python}
class MyClass(Yourclass):
    def __init__(self, my, yours):
        bla = '5 1 2 3 4'
        print bla
\end{python}

\end{solution}
%%
\section{Premiers pas en Python symbolique}

This is an example of theorems.

\subsection{Variable et affection}\index{Variable et affection}

\begin{exercise}
Affectez les variables temps $t$ et distance $d$ par les valeurs 6.892 et 19.7. Calculez et affichez la valeur de la vitesse. Améliorez l’affichage en imposant un chiffre après le point décimal.
\end{exercise}

\begin{solution}
Pour, affectez des variables est les rendre symbolique comme c'est décrit dans le mémo ou il 
sera expliquer temps $t$ et distance $d$ par les valeurs 6.892 et 19.7. Calculez et affichez la 
valeur de la vitesse. Améliorez l’affichage en imposant un chiffre après le point décimal.
\end{solution}

\subsection{Contrôle du flux d’instructions}\index{Theorems!Single Line}
This is a theorem consisting of just one line.

\begin{exercise}
A set $\mathcal{D}(G)$ in dense in $L^2(G)$, $|\cdot|_0$. 
\end{exercise}
\begin{solution}
\end{solution}
%------------------------------------------------

\section{Les Fonctions}\index{Fonctions}

This is an example of a definition. A definition could be mathematical or it could define a concept.

\begin{exercise}
Écrire une fonction cube qui retourne le cube de son argument
\end{exercise}

\begin{exercise}
Écrire une fonction $volumeSphere$ qui calcule le volume d’une sphère de rayon $r$ fourni
en argument et qui utilise la fonction cube .
Tester la fonction $volumeSphere$ par un appel dans le programme principal.
\end{exercise}

\begin{exercise}
Écrire une fonction maFonction qui retourne $f(x) = 2x^{3} + x - 5$
\end{exercise}

\begin{exercise}
Écrire une fonction tabuler avec quatre paramètres : $fonction$ , $borneInf$ , $borneSup$
et $nbPas$ . Cette procédure affiche les valeurs de $fonction$ , de $borneInf$ à $borneSup$ ,
tous les nbPas . Elle doit respecter $borneInf < borneSup$.
Tester cette fonction par un appel dans le programme principal après avoir saisi les
deux bornes dans une floatbox et le nombre de pas dans une integerbox (utilisez le
module easyguiB ).
\end{exercise}

\begin{exercise}
Écrire une fonction $volMasse$ Ellipsoide qui retourne le volume et la masse d’un ellipsoïde grâce à un tuple. Les paramètres sont les trois demi-axes et la masse volumique. On donnera à ces quatre paramètres des valeurs par défaut. \\
On donne: $v = \frac{3}{4} \pi abc$ \\
Tester cette fonction par des appels avec différents nombres d’arguments.
\end{exercise}

\begin{exercise}
Une fonction $f (x)$ est lin\'eaire et a une valeur de $29$ \`a $x = -2$ et $39$ à $x = 3$. Trouver sa valeur à $x = 5$.
\end{exercise}

\begin{exercise}
Pour l'ensemble $N$ de nombres naturels et une opération binaire $f: N x N \longrightarrow N$, on appelle un élément $z$ $\epsilon$ $N$ une identité pour $f$, si $f (a, z) = a = f (z, a)$, pour tout a $\epsilon$ $N$. Lesquelles des opérations binaires suivantes ont une identité?:
\begin{enumerate}
  \item $f (x, y) = x + y - 3$
  \item $f (x, y) = max(x, y)$
  \item $f (x, y) = x^{y}$
\end{enumerate}
\end{exercise}
\begin{solution}
le deuxième et le troisième 
\end{solution}
\section{Structures de données}

\subsection{Logique}
\begin{exercise}
Dans la carte de Karnaugh ci-dessous, $X$ indique un terme sans intérêt. Quelle est la forme minimale de la fonction représentée par la carte de Karnaugh?
\end{exercise}
\subsection{Ensembles}
La notion d'objet immuable en Python est fondamentale,  une structure qui rappel les ensembles en mathématiques que soit fini ou infini est \textit{set}, importante, bien que
dans le cadre de SymPy elle s'appui entièrement sur Python avec certain modification, avec la collection d'objet.
\\

\textit{La fonction set accepte donc en argument un objet de type quelconque et s'efforce de le traduire dans un ensemble. Lorsqu'on ne passe aucun argument à set (option 2), ou qu'on lui passe une liste vide, set renvoie naturellement un ensemble vide; on aurait pu utiliser aussi bien, de la même manière, set(()), set({}), ou même set('') pour arriver au même résultat.}

	\begin{exercise}
		Définir deux ensembles $X = \lbrace a, b, c, d\rbrace$ et  $Y = \lbrace s, b, d\rbrace$ , puis 			affichez les résultats suivants :
 		\begin{enumerate}
  			 \item les ensembles initiaux.
  			 \item le test d’appartenance de l’élément $c$ à $X$.
  			 \item le test d’appartenance de l’élément $a$ à $Y$.
  			 \item les ensembles $X - Y$ et $Y - X$.
  			 \item l’ensemble $X \cup Y$ (union).
  			 \item l'ensemble $X \cap Y$ (intersection).
	 \end{enumerate}
	\end{exercise}

\begin{solution}
Il faut noter qu'il existe une solution qui se base sur le Python builtuints en utilisant la structure de donnée \textit{sets}. Mais comme en n'est dans la logique en utilise 
\begin{python}
from sympy import FiniteSet

X = FiniteSet('a', 'b', 'c', 'd')
Y = FiniteSet('s', 'b', 'd')

class MyClass(Yourclass):
    def __init__(self, my, yours):
        bla = '5 1 2 3 4'
        print bla
\end{python}
\begin{python}
class MyClass(Yourclass):
    def __init__(self, my, yours):
        bla = '5 1 2 3 4'
        print bla
\end{python}

\end{solution}
%\begin{exercise}
%Si $P$, $Q$, $R$ sont des sous-ensembles de l'ensemble universel $U$, alors $(P \cap Q\capR) \cup (P^{c} \cap Q \cap R)\cup(Q^{x}\cupR^{x})$ si
%\begin{enumerate}
%  \item  
%  \item 
%  \item
%\end{enumerate}
%\end{exercise}
%------------------------------------------------
\subsection{Polynômes}
\begin{exercise}
Considérons le polynôme $p(x) = a_{0} + a_{1} x + a_{2} x^{2} + a_{3} x^{3}$, où $a_{i} \neq 0$ $\forall i$. Le nombre minimum de multiplications nécessaires pour évaluer $p$ sur une entrée $x$ est:
\end{exercise}
%-----------------------------------------------
\subsection{Nombres parfaits et nombres chanceux}
\begin{exercise}
$\sqrt{12}$
\end{exercise}
%------------------------------------------------

\section{Remarks}\index{Remarks}

This is an example of a remark.

\begin{remark}
The concepts presented here are now in conventional employment in mathematics. Vector spaces are taken over the field $\mathbb{K}=\mathbb{R}$, however, established properties are easily extended to $\mathbb{K}=\mathbb{C}$.
\end{remark}

%------------------------------------------------

\section{Corollaries}\index{Corollaries}

This is an example of a corollary.

\begin{corollary}[Corollary name]
The concepts presented here are now in conventional employment in mathematics. Vector spaces are taken over the field $\mathbb{K}=\mathbb{R}$, however, established properties are easily extended to $\mathbb{K}=\mathbb{C}$.
\end{corollary}

%------------------------------------------------

\section{Propositions}\index{Propositions}

This is an example of propositions.

\subsection{Several equations}\index{Propositions!Several Equations}

\begin{proposition}[Proposition name]
It has the properties:
\begin{align}
& \big| ||\mathbf{x}|| - ||\mathbf{y}|| \big|\leq || \mathbf{x}- \mathbf{y}||\\
&  ||\sum_{i=1}^n\mathbf{x}_i||\leq \sum_{i=1}^n||\mathbf{x}_i||\quad\text{where $n$ is a finite integer}
\end{align}
\end{proposition}

\subsection{Single Line}\index{Propositions!Single Line}

\begin{proposition} 
Let $f,g\in L^2(G)$; if $\forall \varphi\in\mathcal{D}(G)$, $(f,\varphi)_0=(g,\varphi)_0$ then $f = g$. 
\end{proposition}

%------------------------------------------------

\section{Examples}\index{Examples}

This is an example of examples.

\subsection{Equation and Text}\index{Examples!Equation and Text}

\begin{example}
Let $G=\{x\in\mathbb{R}^2:|x|<3\}$ and denoted by: $x^0=(1,1)$; consider the function:
\begin{equation}
f(x)=\left\{\begin{aligned} & \mathrm{e}^{|x|} & & \text{si $|x-x^0|\leq 1/2$}\\
& 0 & & \text{si $|x-x^0|> 1/2$}\end{aligned}\right.
\end{equation}
The function $f$ has bounded support, we can take $A=\{x\in\mathbb{R}^2:|x-x^0|\leq 1/2+\epsilon\}$ for all $\epsilon\in\intoo{0}{5/2-\sqrt{2}}$.
\end{example}

\subsection{Paragraph of Text}\index{Examples!Paragraph of Text}

\begin{example}[Example name]
\lipsum[2]
\end{example}

%------------------------------------------------

\section{Exercises}\index{Exercises}

This is an example of an exercise.

\begin{exercise}
This is a good place to ask a question to test learning progress or further cement ideas into students' minds.
\end{exercise}

%------------------------------------------------

\section{Problems}\index{Problems}

\begin{problem}
What is the average airspeed velocity of an unladen swallow?
\end{problem}

%------------------------------------------------

\section{Vocabulary}\index{Vocabulary}

Define a word to improve a students' vocabulary.

\begin{vocabulary}[Word]
Definition of word.
\end{vocabulary}

%------------------------------------------------------------------------------
%	CHAPTER 3
%----------------------------------------------------------------------------------------

\chapter{POO}

\section{Programmation Orientée Objet}\index{Notations}
\begin{notation}
Given an open subset $G$ of $\mathbb{R}^n$, the set of functions $\varphi$ are:
\begin{enumerate}
\item Bounded support $G$;
\item Infinitely differentiable;
\end{enumerate}
a vector space is denoted by $\mathcal{D}(G)$. 
\end{notation}

 \subsection{POO}
\begin{exercise}
 Définir une classe Vecteur2D avec un constructeur fournissant les coordonnées par
défaut d’un vecteur du plan (par exemple : $x = 0$ et $y = 0$ ).
Dans le programme principal, instanciez un Vecteur2D sans paramètre, un Vecteur2D
avec ses deux paramètres, et affichez-les.
\end{exercise}
\begin{solution}
 en utilise le module sympy.geometry ce module fait appel à tout les outils et theories qui
 peuvents entre utiliser dans le cade de la géométrie dans le Plan.
 \begin{python}
 from sympy.geometry
  \end{python}
\end{solution}

\begin{exercise}
Enrichissez la classe Vecteur2D précédente en lui ajoutant une méthode d’affichage
et une méthode de surcharge d’addition de deux vecteurs du plan.
Dans le programme principal, instanciez deux Vecteur2D , affichez-les et affichez leur
somme.
\end{exercise}
\begin{solution}
\end{solution}

%------------------------------------------------
\subsection{Notions de COO et d’encapsulation}
%------------------------------------------------

%%
\chapter{Convex Sets}
\section{Convexity}
\subsection{Cone}
\begin{definition}[Cone]
A set $K \in \R^n$, when $x \in K $ implies $\alpha x \in K$.
\end{definition}
A non convex cone can be hyper-plane.\\
For convex cone $x + y \in K, \forall x,y \in K$.\\
Cone don't need to be "pointed". e.g. \\
Direct sums of cones $C_1 + C_2 = \{ x = x_1+x_2 | x_1 \in C_1, x_2 \in C_2 \}$.\\
\begin{example}
$S_1^n  \{ X | X=X^n ,\lambda(x) \ge 0\}$\\
A matrix with positive eigenvalues.
\end{example}

\subsubsection{Operations preserving convexity}
\begin{itemize}
\item[Intersection] $C  \cap_{i \in \mathbb{I}}C_i$
\item[Linear map] Let $A : \mathbb{R}^n \to  \R^n$ be a linear map. If $C \in \R^n$ is convex, so is $A(C) = \{Ax \forall x \in C \}$
\item[Inverse image] $A^{-1}(D) = \{ x \in \R |Ax \in D \}$
\end{itemize}

\subsubsection{Operations that induce convexity}
Convex hull on $S = \cap \{C | S\in C, C is convex\}$\\
\begin{example}
$Co \{ x_1,x_2,\cdots,x_m\} = \{ \sum_{i=1}^m \alpha_i x_i | \alpha \in \delta_m \}$
\end{example}
For a convex set $x \in C \Rightarrow x = \sum \alpha_i x_i$. 
\begin{theorem}[Carathéodory's theorem]
If a point $x \in \R^d$ lies in the convex hull of a set $P$, there is a subset $P'$ of $P$ consisting of $d + 1$ or fewer points such that $x$ lies in the convex hull of $P'$. Equivalently, x lies in an r-simplex with vertices in P.
\end{theorem}

\section{Convex Functions}
\begin{definition}[Convex function]
Let $C \in \R^n$ be convex, $f:C \to \R$ is convex on f if $x,y \in C \times C$. $\forall \alpha \in (0,1)$, $f(\alpha x + (1-\alpha) y) \le f(\alpha x) + f((1-\alpha) y)$
\end{definition}

\begin{definition}[Strictly Convex function]
Let $C \in \R^n$ be convex, $f:C \to \R$ is strictly convex on f if $x,y \in C \times C$. $\forall \alpha \in (0,1)$, $f(\alpha x + (1-\alpha) y) \langle f(\alpha x) + f((1-\alpha) y)$
\end{definition}

\begin{definition}[Strongly convex]
$f:C \to \R$ is strongly convex with modules $u \ge 0$ if $f - \frac{1}{2}u || \cdot ||^2$ is convex.
\end{definition}
Interpretation: There is a convex quadratic $\frac{1}{2}u || \cdot ||^2$ that lower bounds f.
\begin{example}
$\min_{x \in C} f(x) \leftrightarrow \min \bar{f}(x)$
Useful to turn this into an unconstrained problem. \\
$$\bar{f}(x) = \begin{cases}
f(x) \quad if x \in C \\
\infty \quad  elsewhere
\end{cases}$$
\end{example}
\begin{definition}
A function $f : \R^n \to \R \cup \infty \ \bar{\R}$ is convex if $x,y \in \R^n \times \R^n$, $\forall x,y , \bar{f}(\alpha x + (1-\alpha) y) \le f(\alpha x) + f((1-\alpha) y)$
\end{definition}
Definition 1 is equivalent to definition 2 if $f(x) = \infty$.
\begin{example}
$f(x) = \sup_{j \in J} f_j(x)$
\end{example}

\subsection{Epigraph} 
\begin{definition}[Epigraph]
For $f: \R^n \rightarrow \bar{R}$, its epigraph $epi(f) \in \R^{n+1} is the set epi(f) \{ (x,\alpha) | f(x) \in \alpha \}$
\end{definition}
Next: a function is convex i.f.f. its epigraph is convex.

\begin{definition}
A function $f : C \rightarrow \R, C \in \R^n$ is convex if $\forall x, y \in C$, $f(ax + (1-a)x) \le af(x) + (1-a)f(x) \quad \forall a \in (0,1)$.\\ 
Strict convex: $x \neq y \Rightarrow f(ax + (1-a)x) \le af(x) + (1-a)f(x) $
\end{definition}
\begin{remark}
$f$ is convex $\Rightarrow$ $-f$ is concave.
\end{remark}
Level set: $S_{\alpha}f = \{ x | f(x) \le \alpha \}$.\\ 
$S_{\alpha}f$ is convex $\Leftrightarrow$ $f$ is convex. \\
\begin{definition}[Strongly convex]
$f : C \rightarrow \R$ is strongly convex with modules $\mu$ if $\forall x, y \in C$, $\forall \alpha \in (0,1)$, $f(ax + (1-a)x) \le af(x) + (1-a)f(x) - \frac{1}{2\mu}\alpha(1- \alpha) \|x-y\|^2$.
\end{definition}

\begin{remark}
\begin{itemize}
\item $f$ is 2nd-differentiable, $f$ ix \cvx $\iff$ $\nabla^2f(x) \rangle  0$.
\item $f$ is strongly \cvx $\iff$ $\nabla^2f(x) \rangle  \mu I$ $\iff$ $x \ge \mu$
\end{itemize}
\end{remark}
\begin{definition}[2]
$f : \R^n \to \bar{\R} $ is \cvx  if $x, y  \in \R , \alpha \in (0,1), f(ax + (1-a)x) \le af(x) + (1-a)f(x)$.  
\end{definition}
The effective domain of $f$ is $dom f = \{x | f(x) \langle + \infty \}$ 
\begin{example}[ludcator function]
$\delta_c(x) = \begin{cases}
0 \quad  x \in C \\
+ \infty \quad elsewhere
\end{cases}$.\\
$dom \space \delta_c(x) = C$
\end{example}
\begin{definition}[Epigraph]
The epigraph of f is $epi \space f = \{(x,\alpha) | f(x) \le \alpha\}$
\end{definition}
The graph of $epi \space f$ is $\{ (x,f(x) | x \in dom \space f\}$.
\begin{definition}[III]
A function $f : \R^n \to \bar{\R}$ is %\cvx  if $\epi \space f $ is \cvx
\end{definition}
\begin{theorem}
$f : \R^n \to \bar{\R}$ is \cvx  $\iff$ $\forall x,y \in \R^n, \alpha \in (0,1), f(ax + (1-a)x) \le af(x) + (1-a)f(x)$.
\end{theorem}
\begin{proof}
$\Rightarrow$ take $x,y \in dom \space f$, $(x,f(x)) \in epi \space f$,$(y,f(y)) \in epi \space f$.
\end{proof}

\begin{example}[Distance]
Distance to a \cvx  set $d_c(x) = \inf \{ \| z-x \| | z \in C \}$. Take any two sequence $\{ y_k\} and \{ \bar{y}_k\} \subset C$ s.t. $\| y_k - x\| \to d_c(x)$, $\| \bar{y}_k - \bar{x}\| \to d_c(\bar{x})$. $z_k = \alpha y_k + (1 - \alpha) \bar{y}_k$.
\begin{align*}
d_c(\alpha x + (1-\alpha) \bar{x}) &\le \| z_k - \alpha x - (1 - \alpha) \bar{x}\| \\
& = \| \alpha(y_k - x) + (1 - \alpha)(\bar{y}_k - \bar{x})\| \\
& \le \alpha \| y_k - x\| + (1 - \alpha ) \|\bar{y}_k - \bar{x}\|
\end{align*}
Take $k \to \infty$, $d_c(\alpha x + (1 - \alpha) \bar{x}) \le \alpha d(x) + (1 - \alpha) d(\bar{x})$
\end{example}
\begin{example}[Eigenvalues]
Let $X \in S^n := \{ n \times n symmetric matrix\}$. $\lambda_1(x) \ge \lambda_2(X) \ge \ldots \ge \lambda_n(x)$.\\
$f_k(x) = \sum_{1}^n \lambda_i(x)$.\\
Equivalent characterization 

\begin{align*}
f_k(x) & = \max\{ \sum_{i} v_i^T Xv_i | v_i \perp v_j , i \neq j\} \\
& =  \max\{ tr( V^TXV | V^T V = I_k \} \\
\max \{tr(VV^TX) \} \text{by circularity}
\end{align*}
Note $\langle A,B\rangle  = tr(A,B)$ is true for symmetric matrix. \\
$\langle A,A\rangle  = |A |_F^2 = \sum_{i} A_{ii}^2$
\end{example}

\section{Support Function}
Take a set $C \in \R^n$, not necessarily convex.The support function is $\sigma_C = \R^n \to \bar{\R}$. $\sigma_C(x) = \sum \{ \langle x,u\rangle  | u \in C\}$.
\includegraphics[scale=0.5]{1_1.png}
\begin{fact}
The support function binds the supporting hyper-plane.
\end{fact}

Supporting functions are
\begin{itemize}
\item Positively homogeneous\\
$\sigma_C(\alpha x) = \alpha \sigma_C(x) \forall \alpha \rangle  0$ \\
$\sigma_C(\alpha x ) = \sup_{u \in C} \langle \alpha x, u\rangle  = \alpha \sup_{u \in C} \langle x, u\rangle  = \alpha \sigma_C(x)$
\item Sub-linear( a special case of convex, linear combination holds $\forall \alpha$.\\
$\sigma_C(\alpha x + (1 - \alpha) y ) = \sup_{u \in C} \langle \alpha x + (1 - \alpha) y,u\rangle  \le \alpha\sup_{u \in C}\langle x,u\rangle  + (1 - \alpha)\sup_{u \in C}\langle y,u\rangle  $
\end{itemize}
\begin{example}[L2-norm]
$\| x \| = \sup_{u \in C} \{ \langle x, u \rangle, u \in \R^n \}$.\\
$\|x \|_p = \sup \{ \langle x, u \rangle, u \in B_q \}$ where $\frac{1}{p} + \frac{1}{q} = 1$. $B_q = \{ \|x \|_q \le 1\}$.\\
The norm is 
\begin{itemize}
\item Positive homogeneous
\item sub-linear
\item If $0 \in C$, $\sigma_C$ is non-negative.
\item If $C$ is central-symmetric, $\sigma_C(0) = 0$ and $\sigma_C(x) = \sigma_C(-x)$
\end{itemize}
\end{example}

\begin{fact}[Epigraph of a support function]
$epi \space \sigma_C = \{ (x,t) | \sigma_C(x) \le t\}$.
Suppose $(x,t) \in epi \space \sigma_C$. Take any  $\alpha > 0$. $\alpha(x,t) = (\alpha x, \alpha t)$.\\
$\alpha \sigma_C(x) = \alpha \sigma_C(x) \le \alpha t$. $\alpha(x,c) \in epi 
\sigma_C$\\
\includegraphics[]{1_2}
\end{fact}

\section{Operations Preserve Convexity of Functions}
\begin{itemize}
\item Positive affine transformation \\
$f_1,f_2,\ldots,f_k \in \space cvx \R^n$.\\
$f = \alpha_1 f_1 + \alpha_2 f_2 + \ldots + \alpha_k f_k$
\item Supremum of functions. Let $\{ f_i \}_{i \in I}$ be arbitrary family of functions. If $\exists x \sup_{j \in J} f_j(x) < \infty \Leftrightarrow f(x) = \sup_{j \in J} f_j(x) $\\
\includegraphics[]{1_3}
\item Composition with linear map.\\
$f \in cvx \R^n$, $A:\R^n \to \R^m$ is a linear map.
$f \circ A (x) = f(Ax) \in cvx \R^n$\\
\begin{align*}
f \circ A (x) & = f(A(\alpha x + (1-\alpha) y)) \\
& = f(A \alpha x + (1-\alpha) A y) \\
& \le \alpha f(Ax) + (a - \alpha) f(Ay)
\end{align*}
\end{itemize}

%---------------------------------------------------------------------------------------
%	CHAPTER 3
%----------------------------------------------------------------------------------------

\chapter{Nonlinear Problem}
Les sujets de ce chapitre sont du néanmoins axées sur des questions
ou l'approche mathématique et physique et demandé 

\section{Équation différentielle}
Sans aucun doute 
\section{Chaos}\index{Mouvement d'un pendule}
Prenons une pause dans l'apprentissage de nouvelles techniques et algorithmes informatiques
pour un peu, et passer du temps en utilisant ce que nous avons appris jusqu'à présent pour enquêter sur quelque chose d'intéressant. Nous allons commencer avec quelque chose de familier: le simple pendule.
\subsection{Pendule simple}
Le pendule simple figure
\subsection{Pendule à deux bras}
\subsection{Mouvements d’un robot}
Qu'est ce qu'il faut savoir quand en veut modélisé le comportement
d'un robot?. Et bien la réponse est tout simplement des mathématiques

\section{Solution non linéaire d'équation algébrique}\index{Solving Nonlinear Algebraic Equations}

Qu'est ce que non-linéaire et qu'est ce que une \'equation alg\'ebrique

Une \'equation alg\'ebrique est un polyn\^ome de la forme $P(x)$

\begin{equation}
\exp(-x)\sin(x) = \cos(x)
\end{equation}

\section{Transport optimal}
C'est quoi le \textit{ transport optimal}?, exemple simple..., le domaine du transport optimal
est très 
%-------------------------------------------------
\section{Le calcul des variations}

\section{Figure}\index{Figure}

\begin{table}[h]
\centering
\begin{tabular}{l l l}
\toprule
\textbf{Treatments} & \textbf{Response 1} & \textbf{Response 2}\\
\midrule
Treatment 1 & 0.0003262 & 0.562 \\
Treatment 2 & 0.0015681 & 0.910 \\
Treatment 3 & 0.0009271 & 0.296 \\
\bottomrule
\end{tabular}
\caption{Table caption}
\end{table}

%
%\begin{figure}[h]
%\centering\includegraphics[scale=0.5]
%\caption{Figure caption}
%\end{figure}

%----------------------------------------------------------------------------------------
%	Technique Avancée
%----------------------------------------------------------------------------------------

\chapter{Outils avancée}

\section{Programmation Orientée Objet}\index{Notations}
\section{D\'ecorateurs}
Les d\'ecorateurs un m\'ecanisme incontournable pour \'ecrire de tr\'es bon code et purement 
lisible et portable
\subsection{Optimisation du code}
Sans aucun doute l'usage de la programmation symbolique avec ce que en a vue plus haut, ralentisse grandement l'exécution du programme, donc en gagne sur le coté sureté, élégance
et maintenance du code et d'autre part en perd complètement la vitesse; penser à des centaine de ligne de code si vous voulez programmé un robot, voiture ou des objets connectés qui implémente des algorithmes mathématiques et qui de demande beaucoup de ressource est un temps de retour très élevées 

\subsection{Cython}
Cython (http://www.cython.org/ ) est un métalangage qui permet de combiner du code
Python et des types de donn\'ees C, pour concevoir des extensions compilables pour
Python.
Dans un module Cython, il est possible de définir des variables C directement dans
le code Python et de définir des fonctions C qui prennent en paramètre des
variables C ou des objets Python.
Cython contr\^ole ensuite de manière transparente la génération de l’extension C, en
transformant le module en code C par le biais des API C de Python.
Toutes les fonctions Python du module sont alors automatiquement publiées.
Le gain de temps dans la conception introduit par Cython est considérable : toute la
mécanique habituellement mise en œuvre pour créer un module d’extension est
entièrement gérée par Cython.
Ainsi, la fonction max() du module calculs.c pr\'ec\'edemment présent\'ee devient :

Les fichiers Cython ont par convention l’extension pyx, en référence à l’ancien nom.

setup.py pour calculs.pyx

\begin{python}
from distutils.core import setup
from distutils.extension import Extension
from Cython.Distutils import build_ext

extension = Extension("calculs", ["calculs.pyx"])

setup(name="calculs", ext_modules=[extension],cmdclass={'build_ext': build_ext})

\end{python}

\subsection{Theano}
Theano est une bibliothèque pour l'accélération du code lent en Python, tr\'es importante et intéressante
elle offre une syntaxe très particulière.
\section{Interface graphique}
Quelle bibliothèque choisir: sous Python en \`a le choix entre diff\'erente, Tkinter, Gtk, 
Qt, wx et ftk, et il existe encore d'autre bibliothèques qui sont con\c{c}u pour le calcul 
et application scientifique \'editer par Thought[...] dans cette section nous allons. Une 
autre approche serait d'utiliser les ipywdigets avec Jupyter ou JupyterLab notebook, les 
ipywidgets sont trés intéressante approches pour des graphiques interactives rapide d'usages
et bénéfiques sur le plan de présentation par exemple quand en veut exporté ou les partagés 

\chapter{Interface graphique}
Quelles bibliothèque Python pour développé des applications scientifiques graphiques, le choix est 
difficile. D'autant qu'il y en a plusieurs pour ne cité que les plus populaires: Tkinter, Gtk, Qt, wx  
il existe encore d'autre bibliothèques qui sont moins con\c{c}u: Ftk 

Dans cette section nous allons exposés les bibliothèques les plus populaires en mettent l'accent plus particulièrement sur deux d'entre eux: Qt et ipywdigets. 

\end{document}