\chapter{Géométrie plan}
Le module \textcolor{red}{sympy.geometry} permet de créer des entités géométriques bidimensionnelles, telles que des lignes et des cercles, et de rechercher des informations sur ces entités. Cela peut inclure de demander l’aire d’une ellipse, de vérifier la colinéarité d’un ensemble de points ou de rechercher l’intersection de deux lignes. Le cas d'utilisation principal du module implique des entités avec des valeurs numériques, mais il est également possible d'utiliser des représentations symboliques.

\section{Introduction}

\begin{python}
In [1]: from sympy import Point, Polygon, pi
In [2]: p1, p2, p3, p4, p5 = [(0, 0), (1, 0), (5, 1), (0, 1), (3, 0)]
In [3]: Polygon(p1, p2, p3, p4)
Out[3]: Polygon(Point2D(0, 0), Point2D(1, 0), Point2D(5, 1), Point2D(0, 1))
\end{python}

\subsection{Polygones régulier}


\begin{example}
$ABCDEFGH$ est un octogone régulier de centre $O$; calculer $\widehat{ABC}$ (angle entre deux cotés)
\end{example}

\begin{exercise}
Soit $E$ un ensemble de $n \geq 3$ points du plan. On suppose que si $A$ et $B \neq E$, la médiatrice
de $\left[AB\right]$ est un axe de symétrie de $E$. Montrer que $E$ est un polygone régulier.
\end{exercise}

\begin{exercise}
Il est possible de placer trois polygones réguliers (convexes) dans le plan pour qu'ils s'agencent parfaitement autour d'un sommet commun.
\end{exercise}
